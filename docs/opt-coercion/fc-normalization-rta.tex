\documentclass[a4paper,UKenglish]{lipics}

\usepackage{enumerate}
%% \usepackage{abbrev}
\usepackage{xspace}
\usepackage{denot}
\usepackage{prooftree}
\usepackage{afterpage}
\usepackage{float}
%% \usepackage{pstricks}
\usepackage{url}

\usepackage{amsthm}

\usepackage{latexsym} 
%% %% less space consuming enumerates and itemizes
\usepackage{mdwlist} 
\usepackage{stmaryrd} 

%% \usepackage{amsfonts} 
%% \usepackage{amssymb} 
%% % Local packages
\usepackage{code}

%% % \newcommand{\text}[1]{\mbox{#1}}

%% \newcommand{\reach}[2]{\widehat{#1}(#2)}
%% \newcommand{\ftv}[2]{ftv^{#1}(#2)}

\usepackage{color} 
%% %% \newcommand{\color}[1]{}

%% % \newcommand{\scf}{\sigma^\dagger}
%% % \newcommand{\rcf}{\rho^\dagger}
%% % \newcommand{\tcf}{\tau^\dagger}

\newcommand{\simon}[1]{{\bf SPJ:}\begin{color}{blue} #1 \end{color}}
\newcommand{\dv}[1]{{\bf DV:}\begin{color}{red} #1 \end{color}}
\def\fiddle#1{\hspace*{-0.8ex}\raisebox{0.1ex}{$\scriptscriptstyle#1$}}

\newcommand{\highlight}[1]{\colorbox{green}{\ensuremath{#1}}}


\def\twiddleiv{\endprooftree\qquad\prooftree}           % ~~~~
\def\twiddlev{\endprooftree\\ \\ \prooftree}            % ~~~~~...
\def\rulename#1{\textsc{#1}}
\def\minusv#1{\using\text{\rulename{#1}}\justifies}     % \minusv...


\newcommand{\OK}[2]{#1 \vdash^{\fiddle{\sf{E}}} #2}
\newcommand{\wfe}{\vdash^{\fiddle{\sf{E}}}}
\newcommand{\wfco}{\vdash^{\fiddle{\sf{co}}}}
\newcommand{\wftm}{\vdash^{\fiddle{\sf{tm}}}}
\newcommand{\wfty}{\vdash^{\fiddle{\sf{ty}}}}
\newcommand{\unboxed}[1]{\mathop{unboxed}(#1)}

\newcommand{\psim}{\mathrel{\sim_{\tiny \#}}}
\newcommand{\static}{\textsf{$Constraint_{\#}$}}

%% %% \newtheorem{theorem}{Theorem}[section]
%% %% %% \newtheorem{proof}{Proof}[]
%% %% \newtheorem{lemma}[theorem]{Lemma}
%% %% \newtheorem{proposition}[theorem]{Proposition}
%% %% \newtheorem{corollary}[theorem]{Corollary}

%% %% bibliography guidelines 
%% \usepackage{natbib}
%% \bibpunct();A{},
%% \let\cite=\citep

\def\rulename#1{\textsc{#1}}
\def\ruleform#1{\fbox{$#1$}}


%% %%%%%%%%%%%%%%%%%%%%%%%%%%%%%%%%%%%%%%%%%%%%%%%%%%%%%%%%%%%%%%%%%%%%%%
%% % Floats

%% %% \renewcommand{\textfraction}{0.1}
%% %% \renewcommand{\topfraction}{0.95}
%% %% \renewcommand{\dbltopfraction}{0.95}
%% %% \renewcommand{\floatpagefraction}{0.8}
%% %% \renewcommand{\dblfloatpagefraction}{0.8}

%% %% \setlength{\floatsep}{16pt plus 4pt minus 4pt}
%% %% \setlength{\textfloatsep}{16pt plus 4pt minus 4pt}

%% % Figures should be boxed
%% %    *** Uncomment the next two lines to box the floats *** 
\floatstyle{boxed}
\restylefloat{figure}

%% %% % Keep footnotes on one page
%% %% \interfootnotelinepenalty=10000 

%% %% %%%%%%%%%%%%%%%%%%%%%%%%%%%%%%%%%%%%%%%%%%%%%%%%%%%%%%%%%%%%%%%%%%%%%%
%% %% % Indentation
%% %% \setlength{\parskip}{0.35\baselineskip plus 0.2\baselineskip minus 0.1\baselineskip}
%% %% \setlength{\parsep}{\parskip}
%% %% \setlength{\topsep}{0cm}
%% %% \setlength{\parindent}{0cm}

%% \renewcommand{\phi}{\varphi}

\newcommand{\E}{{\cal E}}
\newcommand{\ol}[1]{\overline{#1}}
\newcommand{\sym}[1]{\mathop{sym}\, #1}
\newcommand{\bnd}{\mathop{bnd}}
\newcommand{\cval}{\textsf{cv}}
\newcommand{\nfval}{\textsf{nf}}
\newcommand{\tval}{\textsf{tv}}
\newcommand{\val}{\textsf{val}}
\newcommand{\clift}[1]{\lfloor#1\rfloor}
\newcommand{\lifting}[2]{[#1]{\uparrow}(#2)} 
\newcommand{\erase}[2]{\{#2\}_{#1}}
\newcommand{\nth}[2]{\mathop{nth} #1\;#2}
\newcommand{\inst}[2]{#1@#2}
\newcommand{\refl}[1]{\langle#1\rangle}  % Reflexivity

\newcommand{\tcase}[2]{\mathbf{case}\;#1\;\mathbf{of}\;\ol{#2}}
\newcommand{\tlet}[4]{\mathbf{let}\;#1{:}#2 = #3\;\mathbf{in}\;#4}
\newcommand{\tcast}[2]{#1\;\triangleright\;#2}
\newcommand{\rsa}[1]{\rightsquigarrow_{#1}}
\newcommand{\as}{\ol{a}}
\newcommand{\bs}{\ol{b}}
\newcommand{\cs}{\ol{c}}
\newcommand{\ds}{\ol{d}}
\newcommand{\es}{\ol{e}}
\newcommand{\fs}{\ol{f}}
\newcommand{\gs}{\ol{g}}

\newcommand{\alphas}{\ol{\alpha}}
\newcommand{\betas}{\ol{\beta}}
\newcommand{\gammas}{\ol{\gamma}}
\newcommand{\deltas}{\ol{\delta}}
\newcommand{\epsilons}{\ol{\epsilon}}
\newcommand{\zetas}{\ol{\zeta}}
\newcommand{\etas}{\ol{\eta}}

\newcommand{\phis}{\ol{\phi}}

\newcommand{\sigmas}{\ol{\sigma}}
\newcommand{\taus}{\ol{\tau}}

\newcommand{\xs}{\ol{x}}

%% %% \theoremstyle{plain} 
%% %% \newtheorem{definition}{Definition}[section]


\title{Evidence normalization in System FC}

\author{Dimitrios Vytiniotis}
\author{Simon Peyton Jones}
\affil{Microsoft Research, Cambridge}

   %% \{\{dimitris,simonpj@microsoft.com}}

\authorrunning{D. Vytiniotis and S. Peyton Jones}
% \Copyright{TO BE PROVIDED}

\subjclass{F.4.2 Grammars and Other Rewriting Systems}


\begin{document} 



%% \preprintfooter{\textbf{--- DRAFT submitted to ICFP 2011 ---}}

%% \conferenceinfo{ICFP'08,} {September 22--24, 2008, Victoria, BC, Canada.}
%% \CopyrightYear{2008}
%% \copyrightdata{978-1-59593-919-7/08/09} 



%% \category{D.3.3}{Language Constructs and Features}{Abstract data types}
%% \category{F.3.3}{Studies of Program Constructs}{Type structure}

%% \terms{Design, Languages}

%% \keywords{Type equalities, Deferred type errors, System FC}


%% Theory of computation → Rewrite systems
%% Software and its engineering → Data types and structures


\maketitle
\makeatactive

\begin{abstract}
System FC is an explicitly typed language that serves as the target language for Haskell
source programs. System FC is based on System F with the addition of erasable but explicit type equality
proof witnesses. Equality proof witnesses are generated from type inference performed on source Haskell
programs. Such witnesses may be very large objects, which causes performance degradation in later stages of 
compilation, and makes it hard to debug the results of type inference and subsequent program transformations.
In this paper we present an equality proof simplification algorithm, implemented in GHC, which greatly reduces 
the size of the target System FC programs.
\end{abstract}
%% \category{D.3.3}{Language Constructs and Features}{Abstract data types}
%% \category{F.3.3}{Studies of Program Constructs}{Type structure}
%% \terms{Design,Languages}
%% \keywords{Haskell, Type functions, System FC}

\section{Introduction}\label{s:intro}

A statically-typed intermediate language brings a lot of benefits to a compiler: it is free 
from the design trade-offs that come with source language features; types
can inform optimisations; and type checking
programs in the intermediate language provides a powerful consistency check on 
each stage of the compiler.

%% Type checking the intermediate programs 
%% that result from further program transformation and optimization passes 
%% checks
%% that these stages 
%% the results of further program transformation and optimization passes. 
%% A typed intermediate language provides a firm place for a compiler to stand, 
%% free from the design trade-offs of a complex source language. Moreover, 
%% type-checking the intermediate program provides a 
%% powerful consistency check on the earlier stages of elaboration,
%% desugaring, and optimization.

The Glasgow Haskell Compiler (GHC) has just such an intermediate language,
which has evolved from System F to System FC
\cite{sulzmann+:fc-paper,weirich+:fc2} to accommodate the
source-language features of
\emph{GADTs}~\cite{cheney-hinze:phantom-types,sheard:omega,spj+:gadt}
and \emph{type families}~\cite{Kiselyov09funwith,chak+:synonyms}.
The key feature that allows System FC to accommodate GADTs and type
families is its use of explicit \emph{coercions} that witness the
equality of two syntactically-different types.  Coercions are erased
before runtime but, like types, serve as a static consistency
proof that the program will not ``go wrong''. 

In GHC, coercions are produced by a fairly complex
type inference (and proof inference) algorithm
that elaborates source Haskell programs into FC programs \cite{pjv:modular}. 
Furthermore, coercions undergo major transformations during subsequent program 
optimization passes. As a consequence, they can become very large, 
making the compiler bog down.  This paper describes how we fixed the problem:
\begin{itemize*} 
\item Our main contribution is a novel coercion simplification algorithm, expressed
as a rewrite system, that allows the compiler to replace a coercion
with an equivalent but much smaller one
(Section~\ref{s:normalization}).
     \item Coercion simplification is important in practice.
      We encountered programs whose un-simplified 
      coercion terms grow to many times the size of the actual executable terms, 
      to the point where GHC choked and ran out of heap. When the simplifier
      is enabled, coercions simplify to a small fraction of their
      size (Section~\ref{ssect:ghc}).
     \item To get these benefits, coercion simplification must take user-declared equality axioms 
      into account, but the simplifier {\em must never loop} while optimizing a coercion -- no matter 
      which axioms are declared by users. Proof normalization theorems are notoriously hard, 
      but we present such a theorem for our coercion simplification. (Section~\ref{ssect:termination})
   \end{itemize*}
Equality proof normalization was first studied in the context of monoidal categories and we give 
pointers to early work in Section~\ref{s:related} -- this work in addition addresses the simplification 
of open coercions containing variables and arbitrary user-declared axioms.

%% Coercion simplification has been studied in the context
%% Despite its great practical importance,
%% coercion simplification did not appear to be well-studied in the coercion 
%% literature, but we give some connections to related work in Section~\ref{s:related}.

% --------------------------------------------------------------------------
\section{An overview of System FC} \label{s:intro-coercions}

\begin{figure}\small
\[\begin{array}{l} 
\begin{array}{lrll}
%% \multicolumn{3}{l}{\text{Terms}} \\
c   &  \in & \text{Coercion variables} \\ 
x   & \in  & \text{Term variables} \\ 
e,u & ::= & x \mid l \mid \lambda x{:}\sigma @.@ e \mid e\;u \\ 
    & \mid    & \Lambda a{:}\eta @.@ e \mid e\;\phi   & \text{Type polymorphism} \\
    & \mid    & \lambda c{:}\tau @.@ e \mid e\;\gamma & \text{Coercion abstraction/application} \\
    & \mid    & K \mid \tcase{e}{p \to u} & \text{Constructors and case expressions} \\ 
    & \mid    & \tlet{x}{\tau}{e}{u} & \text{Let binding}    \\ 
    & \mid    & \tcast{e}{\gamma}         &   \text{Cast} \\
%%    & \mid    & \text{\sout{\ensuremath{\lambda c{:}\tau_1 \psim \tau_2 @.@ e}}} & \text{\sout{Coercion abstraction}} \\[2mm]
p   & ::=     & K\;\ol{c{:}\tau}\;\ol{x{:}\tau} & \text{Patterns} \\[3mm]
\end{array}
\end{array}\] 
\caption{Syntax of System FC (Terms)}\label{fig:syntax1}
\end{figure}


We begin by reviewing the role of an intermediate
language.  GHC desugars a rich, complex source language (Haskell) into
a small, simple intermediate language.  The source language, Haskell, is \emph{implicitly typed},
and a type inference engine figures out the type of every binder and sub-expression.
To make type inference feasible, Haskell embodies many somewhat ad-hoc design
compromises; for example, $\lambda$-bound variables are assigned monomorphic types.
By contrast, the intermediate language is simple, uniform, and \emph{explicitly typed}.
It can be typechecked by a simple, linear time algorithm.  The type inference
engine \emph{elaborates} the implicitly-typed Haskell program into an
explicitly-typed FC program.

To make this concrete, Figure~\ref{fig:syntax1} gives the syntax of
System FC, the calculus implemented by GHC's intermediate language.
The term language is mostly conventional, consisting of System F,
together with let bindings, data constructors and case expressions.
The syntax of a term encodes its typing derivation: every binder
carries its type, and type abstractions $\Lambda a{:}\eta@.@e$ and
type applications $e\,\phi$ are explicit. 

\begin{figure}\small
\[\begin{array}{l|l} 
\begin{array}{lrll}
\multicolumn{3}{l}{\text{Types}}               \\ 
\phi,\sigma,\tau,\upsilon & ::= & a        & \text{Variables}   \\ 
        & \mid    & H        & \text{Constants}   \\ 
        & \mid    & F &           \text{Type functions}    \\ 
        & \mid & \phi_1\;\phi_2      & \text{Application} \\ 
%%        & \mid & \phi\;\kappa       & \text{Kind application} \\
        & \mid & \forall a{:}\eta @.@ \phi  & \text{Polymorphic types} \\
%%        & \mid & \forall\kvar @.@ \tau & \text{Kind-polymorphic types}\\[2mm]
\multicolumn{3}{l}{\text{Type constants}}           \\ 
H & ::= & T &           \text{Datatypes}          \\ 
  & \mid    & (\to)   &       \text{Arrow}             \\ 
  & \mid    & (\psim) &  \text{Coercion} \\ 
%%  & \mid    & (\psim) &         \text{Primitive equality type} \\ 
\multicolumn{3}{l}{\text{Kinds}} \\ 
       \kappa,\eta & ::= & \star \mid \kappa \to \kappa \\ 
%%            & \mid   & \forall\kvar @.@ \kappa & \text{Polymorphic kinds} \\
           & \mid   & \static & \text{Coercion kind} \\[2mm]
\end{array} & 
\begin{array}{lrll}
\multicolumn{3}{l}{\text{Coercion values}} \\ 
\gamma,\delta & ::= & c        & \text{Variables} \\ 
%%  & \mid    & \text{\sout{\ensuremath{c}}}             & \text{\sout{Coercion variables}} \\ 
%%  & \mid    & C\;\ol{\kappa}\;\gammas     & \text{Axiom application} \\ 
%%   & \mid    & \gamma_1\;\kappa         & \text{Kind application} \\ 
  & \mid    & \refl{\phi}             & \text{Reflexivity} \\ 
  & \mid    & \gamma_1;\gamma_2        & \text{Transitivity} \\ 
  & \mid    & \sym{\gamma}        & \text{Symmetry} \\ 
  & \mid    & \nth{k}{\gamma}     & \text{Injectivity} \\ 
  & \mid    & \gamma_1\;\gamma_2       & \text{Application} \\
  & \mid    & C\;\gammas          & \text{Type family axiom} \\ 
  & \mid    & \forall a{:}\eta @.@ \gamma   & \text{Polym. coercion} \\
  & \mid    & \inst{\gamma}{\phi}    & \text{Instantiation} \\ \\ \\ \\
%%   & \mid    & \forall\kvar @.@ \gamma & \text{Kind polymorphic coercion} \\ 
%%   & \mid    & \inst{\gamma}{\kappa}     & \text{Kind instantiation}
\end{array} 
\end{array}\]
\caption{Syntax of System FC (types and coercions)}\label{fig:syntax2}
\end{figure}

The types and kinds of the language are given in Figure~\ref{fig:syntax2}. Types include variables ($a$)
and constants $H$ (such as $@Int@$ and @Maybe@), type applications (such as $@Maybe@\;@Int@$), 
and polymorphic types ($\forall a{:}\eta @.@ \phi$). The syntax of types also includes {\em type functions}
(or {\em type families} in the Haskell jargon), which are used to express type level computation.
For instance the following declaration in source Haskell:
\begin{code}
  type family F (a :: *) :: a
  type instance F [a] = a
\end{code}
introduces a type function $F$ at the level of System FC. The accompanying @instance@ line asserts
that any expression of type @F [a]@ can be viewed as having type @a@. We shall see in 
Section~\ref{sec:type-funs} how this fact 
is expressed in FC. Finally type constants include datatype 
constructors ($T$) but also arrow ($\to$) as well as a special type constructor $\psim$ whose 
role we explain in the following section. The kind language includes the familiar $\star$ and
$\kappa_1 \to \kappa_2$ kinds but also a special kind called $\static$ that we explain along with
the $\psim$ constructor.

The typing rules for System FC are given in Figure~\ref{fig:wftm}. We
urge the reader to consult \cite{sulzmann+:fc-paper,weirich+:fc2} for
more examples and intuition. 

\begin{figure}\small
\[\begin{array}{l} 
\begin{array}{lrll}
\multicolumn{3}{l}{\text{Environments}} \\
\Gamma,\Delta  & ::= & \cdot \mid \Gamma,\bnd \\
\bnd & ::=   %% & \kvar & \text{Kind variable} \\
             & a : \eta     & \text{Type variable} \\ 
     & \mid  & c : \sigma \psim \phi    & \text{Coercion variable}\\ 
     & \mid  & x : \sigma   & \text{Term variable}\\ 
     & \mid  & T : \ol{\kappa} \to \star & \text{Data type} \\
     & \mid  & K : \forall (\ol{a{:}\eta}) @.@ \taus \to T\;\as   & \text{Data constructor} \\
     & \mid  & F^n : \ol{\kappa}^n \to \kappa & \text{Type families (of arity $n$)} \\
     & \mid  & C \,\ol{(a{:}\eta)} : \sigma \psim \phi & \text{Axioms} \\
%%     & \mid  & F_n : \kappa_1 \to \ldots \to \kappa_n \to \kappa & \text{Type family, arity $n$} \\
%%     & \mid  & C\;\ol{\kvar}\;(\ol{a{:}\kappa}) :  \peqt{\sigma}{\phi}  & \text{Axiom} \\ 
\multicolumn{3}{l}{\text{Notation}} \\ 
T\;\ol{\tau} & \equiv & T\;\tau_1 \ldots \tau_n \\
\taus \to \tau & \equiv & \tau_1 \to \ldots \to \tau_n \to \tau \\
\taus^{1..n} & \equiv & \tau_1,\ldots,\tau_n
                  %% &  & \text{for $\alpha$ either $\kappa$ or $\tau$}\\
% \Gamma_0       & \equiv & \text{initial (closed) environment}  \\ 
%          & \ni & \multicolumn{2}{l}{(\psim) : \forall\kvar @.@ \kvar \to \kvar \to \static} \\
\end{array}
\end{array}\] 
\caption{Syntax of System FC (Auxiliary definitions) }\label{fig:syntax3}
\end{figure}

\begin{figure}\small
\[\begin{array}{c}\ruleform{\Gamma \wftm e : \tau } \\ \\
\prooftree
   (x{:}\tau) \in \Gamma 
   \minusv{EVar} 
   \Gamma \wftm x : \tau 
   \twiddleiv
   (K{:}\sigma) \in \Gamma
   \minusv{ECon}
   \Gamma \wftm K : \sigma 
   \twiddlev
   \begin{array}{c}
   \Gamma,(x{:}\sigma) \wftm e : \tau \quad
   \Gamma \wfty \sigma : \star \end{array}
   \minusv{EAbs}
   \Gamma \wftm \lambda x{:}\sigma @.@ e : \sigma \to \tau 
   \twiddleiv 
   \begin{array}{c}
   \Gamma \wftm e : \sigma \to \tau \quad \Gamma \wftm u : \sigma 
   \end{array}
   \minusv{EApp}
   \Gamma \wftm e\;u : \tau 
   \twiddlev 
   \begin{array}{c}
   \Gamma,(c{:}\sigma) \wftm e : \tau \\
   \Gamma \wfty \sigma : \static{}
   \end{array}
   \minusv{ECAbs}
   \Gamma \wftm \lambda c{:}\sigma @.@ e : \sigma \to \tau 
   \twiddleiv 
   \begin{array}{c}
   \Gamma \wftm e : (\sigma_1 \psim \sigma_2) \to \tau \\
   \Gamma \wfco \gamma : \sigma_1 \psim \sigma_2
   \end{array}
   \minusv{ECApp}
   \Gamma \wftm e\;\gamma : \tau 
   \twiddlev 
   \begin{array}{c} \phantom{\Gamma} 
%%    \Gamma \wfk \eta \\
   \Gamma,(a{:}\eta) \wftm e : \tau
   \end{array}
   \minusv{ETabs}
   \Gamma \wftm \Lambda a{:}\eta @.@ e : \forall a{:}\eta @.@ \tau 
   \twiddleiv 
   \begin{array}{c} 
   \Gamma \wftm e : \forall a{:}\eta @.@ \tau \quad
   \Gamma \wfty \phi : \eta
   \end{array}
   \minusv{ETApp}
   \Gamma \wftm e\;\phi : \tau[\phi/a]
   \twiddlev 
   %% \begin{array}{c} \phantom{\Gamma} \\ 
   %% \Gamma,\kvar \wftm e : \tau
   %% \end{array}
   %% -----------------------------------------{EKabs}
   %% \Gamma \wftm \Lambda\kvar @.@ e : \forall\kvar @.@ \tau
   %% \twiddleiv 
   %% \begin{array}{c} 
   %% \Gamma \wftm e : \forall\kvar @.@ \tau \\ 
   %% \Gamma \wfk \kappa 
   %% \end{array}
   %% -----------------------------------------{EKApp}
   %% \Gamma \wftm e\;\kappa : \tau[\kappa/\kvar]
   %% \twiddlev
   \begin{array}{c}
     \Gamma,(x{:}\sigma) \wftm u : \sigma \quad
     \Gamma,(x{:}\sigma) \wftm e : \tau 
   \end{array}
   \minusv{ELet}
   \Gamma \wftm \tlet{x}{\sigma}{u}{e} : \tau
%%   \twiddleiv 
%%    \begin{array}{c}
%%      \Gamma \wftm u : \sigma \\
%%      \Gamma,(x{:}\sigma) \wftm e : \tau 
%%    \end{array}
%%    -------------------------------------------{ELet}
%%    \Gamma \wftm \tlet{x}{\sigma}{u}{e} : \tau
   \twiddleiv\hspace{-5pt}
   \begin{array}{c}
    \Gamma \wftm e : \tau \quad
    \Gamma \wfco \gamma : \tau \psim \phi
   \end{array}
   \minusv{ECast} 
   \Gamma \wftm \tcast{e}{\gamma} : \phi
  \twiddlev
   \begin{array}{l} 
   \Gamma \wftm e : T\;\ol{\kappa}\;\sigmas \\ 
   \text{For each branch } K\;\ol{x{:}\tau} \to u \\ 
    \quad (K{:}\forall (\ol{a{:}\eta_a}) @.@ \ol{\sigma_1\psim\sigma_2} \to \taus \to T\;\as) \in \Gamma \\
    \quad \phi_i = \tau_i[\sigmas/\as] \\
    \quad \phi_{1i} = \sigma_{1i}[\sigmas/\as] \\
    \quad \phi_{2i} = \sigma_{2i}[\sigmas/\as]
    \quad \Gamma,\ol{c{:}\phi_1\psim\phi_2}\;\ol{x{:}\phi} \wftm u : \sigma
   \end{array}
   \minusv{ECase}
   \Gamma \wftm \tcase{e}{K\;(\ol{c{:}\sigma_1\psim\sigma_2})\;(\ol{x{:}\tau}) \to u} : \sigma
\endprooftree
\end{array}\]\caption{Well-formed terms}\label{fig:wftm}
\end{figure}



\begin{figure}\small 
\[\begin{array}{c}\ruleform{\Gamma \wfty \tau : \kappa } \\ \\
\prooftree
    (a{:}\eta) \in \Gamma 
  \minusv{TVar}
     \Gamma \wfty a : \eta
  \twiddleiv 
    (T{:}\kappa) \in \Gamma 
  \minusv{TData}
     \Gamma \wfty T : \kappa 
  \twiddleiv 
    (F{:}\kappa) \in \Gamma 
  \minusv{TFun}
     \Gamma \wfty F : \kappa 
  \twiddlev 
   \kappa_1,\kappa_2 \in \{ \static, \star \}
  \minusv{TArr}
   \Gamma \wfty (\to) : \kappa_1 \to \kappa_2 \to \star 
  \twiddleiv
   \phantom{\Gamma}
  \minusv{TEqPred}
   \Gamma \wfty (\psim) : \kappa \to \kappa \to \static
  \twiddlev
  \begin{array}{c}
  \Gamma \wfty \phi_1 : \kappa_1 \to \kappa_2 \quad
  \Gamma \wfty \phi_2 : \kappa_1 
  \end{array} \vspace{2pt}
  \minusv{TApp}
  \Gamma \wfty \phi_1\;\phi_2 : \kappa_2
  \twiddleiv 
  \begin{array}{c} \phantom{\Gamma} \quad
  \Gamma,(a{:}\eta) \wfty \tau : \star 
  \end{array} \vspace{2pt}
  \minusv{TAll}
  \Gamma \wfty \forall a{:}\eta @.@ \tau : \star
\endprooftree
\end{array}\]\caption{Well-formed types}\label{fig:wfty}
\end{figure}

\subsection{Coercions}

The unusual feature of FC is the use of coercions.
The term $\tcast{e}{\gamma}$ is a cast, that converts a term $e$ of
type $\tau$ to one of type $\phi$ (rule \rulename{ECast} in
Figure~\ref{fig:wftm}). The coercion $\gamma$ is a \emph{witness}, 
or \emph{proof}, providing
evidence that $\tau$ and $\phi$ are equal types -- that is, $\gamma$ has
type $\tau \psim \phi$.  
We use the symbol ``$\psim$'' to denote type equality\footnote{The ``$\#$'' subscript
is irrelevant for this paper; the interested reader may consult 
\cite{deferred-type-errors} to understand the related type equality $\sim$, and
the relationship between $\sim$ and $\psim$.}.
The syntax of coercions $\gamma$ is given in
Figure~\ref{fig:syntax2}, and their typing rules in Figure~\ref{fig:wfco}.  
For uniformity we treat $\psim$ as an ordinary type constructor, with
kind $\kappa \to \kappa \to \static$ (Figure~\ref{fig:wfty}). 

To see casts in action, consider this Haskell program which uses GADTs:
\begin{code}
  data T a where               f :: T a -> [a]   
     T1 :: Int -> T Int        f (T1 x) = [x+1]  
     T2 :: a -> T a            f (T2 v) = [v]    

                               main = f (T1 4)   
\end{code}
We regard the GADT data constructor @T1@ as having the type
$$ @T1@ : \forall a. (a \psim @Int@) \to @Int@ \to @T@\;a $$
So in FC, @T1@ takes three arguments: a type argument to instantiate $a$,
a coercion witnessing the equivalence of $a$ and @Int@, and a value of type @Int@.
Here is the FC elaboration of @main@:
\begin{code}
  main = f Int (T1 Int <Int> 4)
\end{code}
The coercion argument has kind $(@Int@\psim@Int@)$, for which the evidence
is just $\refl{@Int@}$ (reflexivity).
Similarly, pattern-matching on @T1@ binds two variables: 
a coercion variable, and a term variable.  Here is the FC elaboration 
of function @f@:
\begin{code}
  f = /\(a:*). \(x:T a).
      case x of 
        T1 (c:a ~# Int) (n:Int) -> (Cons (n+1) Nil) |> sym [c]
        T2 (v:a)                -> Cons v Nil
\end{code}
The cast converts the type of the result from @[Int]@ to @[a]@.
The coercion $\sym{[c]}$ is evidence for (or a proof of) 
the equality of these types, using coercion @c@, of type $(@a@\psim@Int@)$.

\subsection{Typing coercions} \label{sec:newtype} \label{sec:type-funs}

\begin{figure*}\small
\[\begin{array}{c}\ruleform{\Gamma \wfco \gamma : \sigma_1 \psim \sigma_2 } \\ \\
\prooftree
    \begin{array}{c} \phantom{G} \\ 
    (c{:}\sigma_1 \psim \sigma_2) \in \Gamma
    \end{array}
    \minusv{CVar}
    \Gamma \wfco c : \sigma_1 \psim \sigma_2 
    \twiddleiv 
    \begin{array}{c}
      (C\, \ol{a{:}\eta} : \tau_1 \psim \tau_2) \in \Gamma \\ 
    \Gamma \wfco \gamma_i : \sigma_i \psim \phi_i \vspace{1pt}
    \end{array}
    \minusv{CAx} 
    \Gamma \wfco C\;\gammas : \tau_1[\sigmas/\as]{\psim}\tau_2[\phis/\as]
    \twiddleiv 
    \begin{array}{c} \phantom{G} \\ 
    \Gamma \wfty \phi : \kappa
    \end{array}
    \minusv{CRefl}
    \Gamma \wfco \refl{\phi} : \sigma \psim \sigma
    \twiddlev 
    \begin{array}{c} 
      \Gamma \wfco \gamma_1 : \sigma_1 \psim \sigma_2 \\ 
      \Gamma \wfco \gamma_2 : \sigma_2 \psim \sigma_3 \vspace{1pt}
    \end{array}
    \minusv{CTrans}
    \Gamma \wfco \gamma_1;\gamma_2 : \sigma_1{\psim}\sigma_3
    \twiddleiv
    \Gamma \wfco \gamma : \sigma_1 \psim \sigma_2 
    \minusv{CSym} 
    \Gamma \wfco \sym{\gamma} : \sigma_2 \psim \sigma_1 
    \twiddleiv 
    \Gamma \wfco \gamma : H\;\sigmas \psim H\;\taus
    \minusv{CNth} 
    \Gamma \wfco \nth{k}{\gamma} : \sigma_k \psim \tau_k 
    \twiddlev 
    \Gamma,(a{:}\eta) \wfco \gamma : \sigma_1 \psim \sigma_2
    \minusv{CAll} 
    \Gamma \wfco \forall a{:}\eta @.@ \gamma : (\forall a{:}\eta @.@ \sigma_1) \psim (\forall a{:}\eta @.@ \sigma_2)
    %% \twiddleiv
    %% \Gamma \wfco \gamma_i : \tau_i \psim \sigma_i \quad i \in 1..n
    %% \minusv{CFun} 
    %% \Gamma \wfco F^n\;\gammas : F\;\taus \psim F\;\sigmas
    \twiddlev
    \begin{array}{c}
      \Gamma \wfco \gamma_1 : \sigma_1 \psim \sigma_2  \\
      \Gamma \wfco \gamma_2 : \phi_1 \psim \phi_2 \quad \Gamma \wfty : \sigma_1\;\phi_1 : \kappa
    \end{array}
    \minusv{CApp}
    \Gamma \wfco \gamma_1\;\gamma_2 : \sigma_1\;\phi_1 \psim \sigma_2\;\phi_2
    \twiddleiv 
    \begin{array}{c}
    \Gamma \wfty \phi : \eta \\
    \Gamma \wfco \gamma : (\forall a{:}\eta @.@ \sigma_1) \psim (\forall a{:}\eta @.@ \sigma_2) \vspace{1pt}
    \end{array}
    \minusv{CInst} 
    \Gamma \wfco \inst{\gamma}{\phi} : \sigma_1[\phi/a] \psim \sigma_2[\phi/a]
\endprooftree
\end{array}\]\caption{Well-formed coercions}\label{fig:wfco}
\end{figure*}

Figure~\ref{fig:wfco} gives the typing rules for coercions. The rules include unsurprising cases 
for reflexivity (\rulename{CRefl}), symmetry (\rulename{CSym}), and transitivity (\rulename{CTrans}).
Rules \rulename{CAll} and \rulename{CApp} allow us to construct coercions on more complex types from
coercions on simpler types. Rule \rulename{CInst} instantiates a coercion between two $\forall$-types,
to get a coercion between two instantiated types. Rule \rulename{CVar} allows us to use a coercion 
that has been introduced to the context by a coercion abstraction $(\lambda c{:}\tau{\psim}\phi @.@ e)$, 
or a pattern match against a GADT (as in the example above).

Rule \rulename{CAx} refers to instantiations of {\em axioms}. In GHC, axioms can arise as a result of {\em newtype} or {\em type family} declarations. Consider the following code:
\begin{code}
  newtype N a = MkN (a -> Int)

  type family F (x :: *) :: *
  type instance F [a]  = a
  type instance F Bool = Char
\end{code}
$N$ is a \emph{newtype} (part of the original Haskell 98 definition), and is desugared to 
the following FC coercion axiom:
\[\begin{array}{rcl}
C_N \, a& : & N\,a \psim a \rightarrow @Int@ 
\end{array}\] 
which provides evidence of the equality of types $(N\,a)$ and $(a \rightarrow @Int@)$.

In the above Haskell code, $F$ is a {\em type family} \cite{chak+:types, chak+:synonyms}, 
and the two @type@ @instance@ declarations above introduce two FC coercion axioms: 
\[\begin{array}{rcl}
C_1 \, a & : & F\;[a] \psim a \\
C_2 & : & F\;@Bool@ \psim @Char@
\end{array}
\]
Rule \rulename{CAx} describes how these axioms may be used to create coercions. In this 
particular example, if we have $\gamma : \tau \psim \sigma$, then we can prove that
$ C_1\;\gamma : F\;[\tau] \psim \sigma$. Using such coercions we can get, for example, that
$(\tcast{3}{\sym{(C_1\;\refl{@Int@})}}) : F\;[@Int@]$.

Axioms always appear saturated in System FC, hence the syntax $C\,\overline{\gamma}$ in Figure~\ref{fig:syntax2}.

%% Lifting and one push rule, in case we need it 
%% \begin{figure*}\small
%% \[\begin{array}{c}
%% \ruleform{\lifting{a\mapsto\gamma}{\tau} = \gamma'} \\ \\
%% \begin{array}{lcl}
%% \lifting{a \mapsto \gamma}{a}    & = & \gamma \\
%% \lifting{a \mapsto \gamma}{b}    & = & \refl{b} \\ 
%% \lifting{a \mapsto \gamma}{H}    & = & \refl{H} \\
%% \lifting{a \mapsto \gamma}{F}    & = & \refl{F} \\
%% \lifting{a \mapsto \gamma}{\tau_1\;\tau_2} & = & 
%%    \left\{\begin{array}{l} 
%%            \refl{\phi_1\;\phi_2} \text{ when } \lifting{a\mapsto\gamma}{\tau_i} = \refl{\phi_i} \\
%%            (\lifting{a \mapsto \gamma}{\tau_1})\;(\lifting{a\mapsto\gamma}{\tau_2}) \text{ otherwise }
%%          \end{array}\right.   \\
%% \lifting{a\mapsto\gamma}{\forall a{:}\eta @.@ \tau} & = &  
%%    \left\{\begin{array}{l} 
%%            \refl{\forall a{:}\eta @.@ \phi} \text{ when } \lifting{a\mapsto\gamma}{\tau} = \refl{\phi} \\
%%            \forall a{:}\eta @.@ (\lifting{a\mapsto\gamma}{\tau}) \text{ otherwise}
%%          \end{array}\right.
%% \end{array} \\ \\ 
%% \begin{array}{llcl} 
%%         & \tcase{\tcast{K\;\taus\;\gammas\;\es}{\gamma}}{p \to u} & \rightsquigarrow & \tcase{K\;\taus'\;\gammas'\;\es'}{p \to u} \\ 
%%                    & & & \hspace{-4pt}\begin{array}{lll} 
%%                                   \text{when } \\ %% & \cval(K\;\taus\;\phis\;\es)  \\ 
%%                                   & \wfco \gamma : T\;\taus \psim T\;\taus' \\ 
%%                                   & K{:}\forall\ol{a{:}\eta} @.@ \ol{\sigma_1 \psim \sigma_2} \to \sigmas \to T\;\as \in \Gamma_0 \\ 
%%                                   & e_i' = \tcast{e_i}{\lifting{\as \mapsto \deltas}{\sigma_i[\phis/\cs]}} \\
%%                                   & \delta_j = \nth{j}{\gamma} \\ 
%%                                   & \gamma_j' = \ldots 
%%                      \end{array}
%% \end{array}
%% \end{array} \]
%% \caption{Coercion pushing in simplification}\label{fig:opsem}
%% \end{figure*}


% --------------------------------------------------------------------------
\section{The problem with large coercions}\label{ssect:large}

System FC terms arise as the result of elaboration of source language
terms, through type inference. Type inference typically
relies on a \emph{constraint solver} \cite{pjv:modular}
which produces System FC witnesses of
equality (coercions), that in turn decorate the elaborated term. The constraint solver is
not typically concerned with producing small or readable witnesses;
indeed GHC's constraint solver can produce large and complex coercions.
These complex coercions can make the
elaborated term practically impossible to understand and debug.

Moreover, GHC's optimiser transforms well-typed FC terms.
Insofar as these transformations involve coercions, the coercions \emph{themselves}
may need to be transformed.  If you think of the coercions as little proofs that
fragments of the program are well-typed, then the optimiser must maintain the proofs
as it transforms the terms.  

\subsection{How big coercions arise}

The trouble is that \emph{term-level optimisation tends to make
coercions bigger}. The full details of these transformations are given in the so called {\em push}
rules in our previous work~\cite{weirich+:fc2}, but we illustrate them here with an example.
Consider this term:
$$
  (\tcast{\lambda x.e}{\gamma})\, a
$$
where
$$
\begin{array}{rcl}
  \gamma & : & (\sigma_1 \rightarrow \tau_1) \psim (\sigma_2 \rightarrow \tau_2) \\
  a      & : & \sigma_2
\end{array}
$$
We would like to perform the beta reduction, but the cast is getting in
the way.  No matter!  We can transform thus:
$$\begin{array}{ll}
  &  (\tcast{\lambda x.e}{\gamma})\, a \\
= &  \tcast{((\lambda x.e)\, (\tcast{a}{\sym{(\nth{0}{\gamma})}}))}{\nth{1}{\gamma}}
\end{array}
$$
From the coercion $\gamma$ we have derived two coercions whose syntactic form
is larger, but whose types are smaller:
$$
\begin{array}{rcl}
  \gamma & : & (\sigma_1 \rightarrow \tau_1) \psim (\sigma_2 \rightarrow \tau_2) \\
\sym{(\nth{0}{\gamma})} & : & \sigma_2 \psim \sigma_1  \\
\nth{1}{\gamma} & : & \tau_1 \psim \tau_2 
\end{array}
$$
Here we make use of the coercion combinators $sym$, which reverses the sense of
the proof; and $nth\,i$, which from a proof of $T\,\overline{\sigma} \psim T \, \overline{\tau}$
gives a proof of $\sigma_i \psim \tau_i$.  Finally, we use the derived coercions to 
cast the argument and result of the function separately.  Now the lambda is
applied directly to an argument (without a cast in the way), so 
$\beta$-reduction can proceed as desired.
Since $\beta$-reduction is absolutely
crucial to the optimiser, this ability to ``push coercions out of the way'' is
fundamental. Without it, the optimiser is hopelessly compromised.

A similar situation arises with @case@ expressions:
$$@case@\,(\tcast{K\,e_1}{\gamma})\,@of@\,\{\ldots;\,K\,x \rightarrow e_2; \ldots \}$$
where $K$ is a data constructor.  
Here we want to simplify the @case@ expression, by picking the correct alternative
$K\,x \rightarrow e_2$, and substituting $e_1$ for $x$.  Again the coercion gets in the way, but
again it is possible to push the coercion out of way.  

\subsection{How coercions can be simplified}

Our plan is to simplify complicated coercion terms into simpler ones, using rewriting. 
Here are some obvious rewrites we might think of immediately:
$$
\begin{array}{rcll}
\sym{(\sym{\gamma})} & \rsa{} & \gamma \\
\gamma ; \sym{\gamma} & \rsa{} & \refl{\tau} & \text{if}\,\gamma : \tau \psim \phi 
\end{array}
$$
But there are much more complicated rewrites to consider.
Consider these coercions, where $C_N$ is the axiom generated by the newtype coercion in 
Section~\ref{sec:newtype}:
$$
\begin{array}{rcl}
\gamma_1 & : & \tau_1 \psim \tau_2 \\
\gamma_2 = \sym{(C_N\,\refl{\tau_1})} & : & (\tau_1 \rightarrow @Int@) \psim (N\,\tau_1) \\
\gamma_3 = N\,\refl{\gamma_1} & : & (N\,\tau_1) \psim (N\,\tau_2) \\
\gamma_4 = C_N\,\refl{\tau_2} & : & (N\,\tau_2) \psim (\tau_2 \rightarrow @Int@) \\
\\
\gamma_5 = \gamma_2 ; \gamma_3 ; \gamma_4 & : & (\tau_1 \rightarrow @Int@) \psim (\tau_2 \rightarrow @Int@)
\end{array}
$$
Here $\gamma_2$ takes a function, and wraps it in the newtype; then $\gamma_3$ coerces that newtype from
$N\,\tau_1$ to $N\,\tau_2$; and $\gamma_4$ unwraps the newtype.
Composing the three gives a rather large, complicated 
coercion $\gamma_2 ; \gamma_3 ; \gamma_4$.  \emph{But its type
is pretty simple}, and indeed the coercion $\gamma_1 \to \refl{@Int@}$ is a much simpler
witness of the same equality.  The rewrite system we present shortly will rewrite 
the former to the latter.

Finally, here is an actual example taken from a real program compiled by GHC
(don't look at the details!):
$$
\begin{array}{ll}
& @Mut@\, \refl{v}\, (\sym{(C_{StateT} \, \refl{s})})\, \refl{a} \\
& ; \sym{(\nth{0}{(\inst{\inst{\inst{(\forall 
    w % :*\rightarrow *
    t % :*
    b % : *
.\,
                   @Mut@\, \refl{w}\, (\sym{(C_{StateT}\, \refl{t})})\, \refl{b}
                   \rightarrow \refl{@ST@\, t\,(w\, b)})}{v}}{s}}{a})}} \\
\rsa{} & \refl{@Mut@\, v\,s\,a}
\end{array}
$$
As you can see, the shrinkage in coercion size can be dramatic.

%% \dv{Simon will update some of the examples here. Not sure if we should
%%   include the push figure then if we can explain it with a couple of
%%   examples.  The figure will introduce lifting etc and maybe that is a
%%   distraction. However Simon, note that we refer to this example from
%%   a later section and describe how exactly it was optimized, so probably 
%%   you do not want to eliminate it entirely, but just add more examples?}


\section{Coercion simplification}\label{s:normalization}
\newcommand{\G}{{\cal G}}

We now proceed to the details of our coercion simplification algorithm. We note that the design of the algorithm
is guided by empirical evidence of its effectiveness on actual programs and that other choices might be possible.
Nevertheless, we formally study the properties of this algorithm, namely we will show that it preserves validity 
of coercions and terminates -- even when the rewrite system induced by the axioms is not strongly normalizing.

\subsection{Simplification rules}\label{ssect:rules}

Coercion simplification is given as a non-deterministic relation in Figure~\ref{fig:optimization1} and Figure~\ref{fig:optimization2}
In these two figures we use some syntactic conventions: Namely, for sequences of coercions $\gammas_1$ and $\gammas_2$, 
we write $\ol{\gamma_1;\gamma_2}$ for the sequence of pointwise transitive compositions and $\sym{\gammas_1}$ for pointwise 
application of symmetry. We write $nontriv(\gamma)$ iff $\gamma$ {\em contains} some variable $c$ or axiom application $C\;\gammas$.

\begin{figure}\small
\[\begin{array}{l}
\text{Coercion evaluation contexts} \quad\quad \G ::=  \Box \mid \G\;\gamma \mid \gamma\;\G \mid C\;\gammas_1\G\gammas_2 \mid  \sym{\G} \mid \forall a{:}\eta @.@ \G \mid \inst{\G}{\tau} \mid \G;\gamma \mid \gamma;\G \\ \\ 
\prooftree
     \begin{array}{c}
        \gamma  \approx \G[\gamma_1] \text{ modulo associativity of } ({;}) \quad
        \Delta \wfco \gamma_1 : \sigma \psim \phi \quad \Delta \vdash \gamma_1 \rsa{} \gamma_2 \vspace{2pt}
     \end{array}
     \minusv{CoEval}
      \gamma \longrightarrow \G[\gamma_2] 
\endprooftree \\ \\ 
\ruleform{\Delta \vdash \gamma_1 \rsa{} \gamma_2} \\ \\ 
\begin{array}{llcl}
\multicolumn{4}{l}{\text{Reflexivity rules}} \\ 
\rulename{ReflApp}    & \Delta \vdash \refl{\phi_1}\;\refl{\phi_2} & \rightsquigarrow & \refl{\phi_1\;\phi_2} \\
\rulename{ReflAll}    & \Delta \vdash \forall a{:}\eta @.@ \refl{\phi}   & \rightsquigarrow & \refl{\forall a{:}\eta @.@ \phi} \\
\rulename{ReflElimL}  & \Delta \vdash \refl{\phi};\gamma & \rsa{} & \gamma \\ 
\rulename{ReflElimR}  & \Delta \vdash \gamma;\refl{\phi} & \rsa{} & \gamma \\ \phantom{\Delta}
\end{array} \\  
\begin{array}{llcl} 
\multicolumn{4}{l}{\text{Eta rules}} \\ 
\rulename{EtaAllL}  & \Delta \vdash \inst{((\forall a{:}\eta @.@ \gamma_1);\gamma_2)}{\phi} & \rsa{} & \gamma_1[\phi/a];(\inst{\gamma_2}{\phi}) \\
\rulename{EtaAllR}  & \Delta \vdash \inst{(\gamma_1;(\forall a{:}\eta @.@ \gamma_2))}{\phi} & \rsa{} & \inst{\gamma_1}{\phi};\gamma_2[\phi/a] \\
\rulename{EtaNthL}  & \Delta \vdash \nth{k}{(\refl{H\;\taus^{1..\ell}}\;\gammas;\gamma)}      & \rsa{} & \left\{\begin{array}{ll} \nth{k}{\gamma}     & \text{ if } k \leq \ell  \\ 
                                                                                                            \gamma_{k-\ell};\nth{k}{\gamma} & \text{ otherwise }
                                                                                                            
                                                                                          \end{array}\right. \\ 
\rulename{EtaNthR}  & \Delta \vdash \nth{k}{(\gamma;\refl{H\;\taus^{1..\ell}}\;\gammas)}      & \rsa{} & \left\{\begin{array}{ll}
                                                                                                     \nth{k}{\gamma}        & \text{ if } k \leq \ell \\  
                                                                                                     \nth{k}{\gamma};\gamma_{k-\ell} & \text{ otherwise }
                                                                              \end{array}\right.
\end{array} \\ 
\begin{array}{llcl}
\multicolumn{4}{l}{\text{Symmetry rules}} \\
\rulename{SymRefl}  & \Delta \vdash \sym{\refl{\phi}} & \rightsquigarrow & \refl{\phi} \\ 
\rulename{SymAll}   & \Delta \vdash \sym{(\forall a{:}\eta @.@ \gamma)} & \rightsquigarrow & \forall a{:}\eta @.@ \sym{\gamma} \\ 
\rulename{SymApp}   & \Delta \vdash \sym{(\gamma_1\;\gamma_2)} & \rightsquigarrow & (\sym{\gamma_1})\;(\sym{\gamma_2}) \\ 
\rulename{SymTrans} & \Delta \vdash \sym{(\gamma_1;\gamma_2)}  & \rightsquigarrow & (\sym{\gamma_2}){;}(\sym{\gamma_1}) \\ 
\rulename{SymSym}   & \Delta \vdash \sym{(\sym{\gamma})} & \rightsquigarrow & \gamma 
\end{array}  \\ \\
\begin{array}{llcl}
\multicolumn{4}{l}{\text{Reduction rules}} \\
\rulename{RedNth}   & \Delta \vdash \nth{k}{(\refl{H\;\taus^{1..\ell}}\;\gammas)} & \rightsquigarrow & \left\{\begin{array}{ll} \refl{\tau_k} & \text{ if }k \leq \ell \\ 
                                                                                                               \gamma_{k-\ell}          & \text{ otherwise } 
                                                                                                \end{array}\right. \\ 
\rulename{RedInstCo}  & \Delta \vdash \inst{(\forall a{:}\eta @.@ \gamma)}{\phi} & \rsa{} & \gamma[\phi/a] \\  
\rulename{RedInstTy}  & \Delta \vdash \inst{\refl{\forall a{:}\eta @.@ \tau}}{\phi} & \rsa{} & \refl{\tau[\phi/a]}
\end{array} \\ \\ 
\begin{array}{llcll}
\multicolumn{4}{l}{\text{Push transitivity rules }} \\
\rulename{PushApp} & \Delta \vdash (\gamma_1\;\gamma_2);(\gamma_3\;\gamma_4) & \rsa{} & (\gamma_1;\gamma_3)\;(\gamma_2;\gamma_4) \\ 
\rulename{PushAll} & \Delta \vdash (\forall a{:}\eta @.@ \gamma_1); (\forall a{:}\eta @.@ \gamma_2) & \rsa{} & \forall a{:}\eta @.@ \gamma_1;\gamma_2 \\ 
\rulename{PushInst}& \Delta \vdash (\inst{\gamma_1}{\tau});(\inst{\gamma_2}{\tau}) & \rsa{} & \inst{(\gamma_1;\gamma_2)}{\tau}  
                                                              & \text{ when } \Delta \wfco \gamma_1;\gamma_2 : \sigma_1 \psim \sigma_2 \\ 
\rulename{PushNth} & \Delta \vdash (\nth{k}{\gamma_1});(\nth{k}{\gamma_2})   & \rsa{} & \nth{k}{(\gamma_1;\gamma_2)}
                                                               & \text{ when } \Delta \wfco \gamma_1;\gamma_2 : \sigma_1 \psim \sigma_2 
\end{array}
\end{array}\]\caption{Coercion simplification (I)}\label{fig:optimization1}
\end{figure}

We define coercion evaluation contexts, $\G$, as coercion terms with holes inside them. The syntax of $\G$ allows us to rewrite anywhere 
inside a coercion. The main coercion evaluation rule is \rulename{CoEval}. If we are given a coercion $\gamma$, we first decompose it to some 
evaluation context $\G$ with $\gamma_1$ in its hole. Rule \rulename{CoEval} works up to associativity of transitive composition;
% -- for example, the compiler may use a flat list of coercions 
%that are transitively composed to each other instead of the binary composition operator $(;)$. 
for example, we will 
allow the term $(\gamma_1;\gamma_2;);\gamma_3$ to be written as $\G[\gamma_2;\gamma_3]$ where $\G = \gamma_1;\Box$. This treatment of 
transitivity is extremely convenient, but we must be careful to ensure that our argument for termination 
remains robust under associativity (Section~\ref{ssect:termination}). Once we
have figured out a decomposition $\G[\gamma_1]$, \rulename{CoEval} performs 
a single step of rewriting $\Delta \vdash \gamma_1 \rsa{} \gamma_2$ and simply return $\G[\gamma_2]$. 
Since we are allowed to rewrite coercions under a type environment ($\forall a{:}\eta @.@ \G$ is a valid coercion 
evaluation context), $\Delta$ (somewhat informally) enumerates the type variables bound by $\G$. For instance we
should be allowed to rewrite $\forall a{:}\eta @.@ \gamma_1$ to $\forall a{:}\eta @.@ \gamma_2$. This can happen 
if $(a{:}\eta) |- \gamma_1 \rsa{} \gamma_2$. The precondition $\Delta \wfco \gamma_1 : \sigma \psim \phi$ of rule
\rulename{CoEval} ensures that this context corresponds to the decomposition of $\gamma$ into a context and $\gamma_1$.
Moreover, the $\Delta$ is passed on to the $\rsa{}$ relation, since some of the rules of the $\rsa{}$ relation that we will present 
later may have to consult the context $\Delta$ to establish preconditions for rewriting.

The soundness property for the $\longrightarrow$ relation is given by the following theorem.
\begin{theorem}[Coercion subject reduction]\label{thm:sr-theorem}
If $\wfco \gamma_1 : \sigma \psim \phi$ and $\gamma_1 \longrightarrow \gamma_2$ then $\wfco \gamma_2 : \sigma \psim \phi$. 
\end{theorem}
The rewriting judgement $\Delta \vdash \gamma_1 \rsa{} \gamma_2$ satisfies a similar property. 
\begin{lemma}\label{lem:sr-lemma}
If $\Delta \wfco \gamma_1 : \sigma \psim \phi$ and $\Delta \vdash \gamma_1 \rsa{} \gamma_2$ then $\Delta \wfco \gamma_2 : \sigma \psim \phi$. 
\end{lemma}

To explain coercion simplification, we now present the reaction rules 
for the $\rsa{}$ relation, organized in several groups.

\subsubsection{Pulling reflexivity up} 
Rules \rulename{ReflApp}, \rulename{ReflAll}, \rulename{ReflElimL}, and \rulename{ReflElimR}, deal with 
uses of reflexivity. Rules \rulename{ReflApp} and \rulename{ReflAll} ``swallow'' constructors from the 
coercion language (coercion application, and quantification respectively) into the type language 
(type application, and quantification respectively). Hence they pull reflexivity as high as 
possible in the tree structure of a coercion term. Rules \rulename{ReflElimL} and \rulename{ReflElimR} 
simply eliminate reflexivity uses that are composed with other coercions. 

\subsubsection{Pushing symmetry down} 
Uses of symmetry, contrary to reflexivity, are pushed as close to the leaves as possible or eliminated, 
(rules \rulename{SymRefl}, \rulename{SymAll}, \rulename{SymApp}, \rulename{SymTrans}, and \rulename{SymSym})
only getting stuck at terms of the form 
$\sym{x}$ and $\sym{(C\;\gammas)}$.
The idea is that by pushing uses of symmetry towards the leaves, 
the rest of the rules may completely ignore symmetry, except where 
symmetry-pushing gets stuck (variables or axiom applications). 

\subsubsection{Reducing coercions}
Rules \rulename{RedNth}, \rulename{RedInstCo}, and \rulename{RedInstTy} comprise the first interesting group of rules.
They eliminate uses of injectivity and instantiation. Rule \rulename{RedNth} is concerned with the case where 
we wish to decompose a coercion of type $H\;\phis \psim H\;\sigmas$, where the coercion term contains $H$ in its head.
Notice that $H$ is a type and may already be applied to some type arguments $\taus^{1..\ell}$, and hence the rule 
has to account for selection from the first $\ell$ arguments, or a later argument. Rule \rulename{RedInstCo} deals
with instantiation of a polymorphic coercion with a type. Notice that in rule \rulename{RedInstCo} the quantified variable
may only appear ``protected'' under some $\refl{\sigma}$ inside $\gamma$, and hence simply substituting $\gamma[\phi/a]$ is 
guaranteed to produce a syntactically well-formed coercion. Rule \rulename{RedInstTy} deals with the instantiation of a 
polymorphic coercion that is {\em just} a type.

\subsubsection{Eta expanding and subsequent reducing}
Redexes of \rulename{RedNth} and \rulename{RedInstCo} or \rulename{RedInstTy} may not be directly visible. 
Consider $\nth{k}{(\refl{H\;\taus^{1..\ell}}\;\gammas;\gamma)}$. The use of transitivity stands in our way for the 
firing of rule \rulename{RedNth}. To the rescue, rules \rulename{EtaAllL}, \rulename{EtaAllR}, \rulename{EtaNthL}, 
and \rulename{EtaNthR}, push decomposition or instantiation through transitivity and eliminate such redexes. 
We call these rules ``eta'' because in effect we are $\eta$-expanding and immediately reducing one of the components of the transitive composition. 
Here is a decomposition of \rulename{EtaAllL} in smaller steps that involve an $\eta$-expansion (of $\gamma_2$ in the second line): 
\[\begin{array}{ll}
        & \inst{((\forall a{:}\eta @.@ \gamma_1);\gamma_2)}{\phi} \\
   \rsa{} & \inst{((\forall a{:}\eta @.@ \gamma_1);(\forall a{:}\eta @.@ \inst{\gamma_2}{a}))}{\phi}          \\ 
 \rsa{} & \inst{(\forall a{:}\eta @.@ \gamma_1;\inst{\gamma_2}{a})}{\phi} \;\;\rsa{}\;\; \gamma_1[\phi/a] ; \inst{\gamma_2}{\phi} 
\end{array}\]
We have merged these steps in a single rule to facilitate the proof of 
termination. In doing this, we do not lose any reactions, since all of the intermediate terms can also reduce to the final coercion.

There are many design possibilities for rules that look like our $\eta$-rules. For instance one may wonder why 
we are not always expanding terms of the form $\gamma_1;(\forall a{:}\eta @.@ \gamma_2)$ to $\forall a{:}\eta @.@ \inst{\gamma_1}{a} ; \gamma_2$, 
whenever $\gamma_1$ is of type $\forall a{:}\eta @.@ \tau \psim \forall a{:}\eta @.@ \phi$. We experimented with several variations like this, but we found 
that such expansions either complicated the termination argument, or did not result in smaller coercion terms. Our rules in 
effect perform $\eta$-expansion {\em only} when there is a firing reduction directly after the expansion. 

%% Finally, one may wonder if we are missing a rule that reduces $\inst{(\gamma_1;(\forall a{:}\eta @.@ \gamma_2);\gamma_3)}{\phi}$ (and similarly for $\mathrel{nth}$). 
%% However, if $\gamma_1$ or $\gamma_3$ are $\forall$-coercions then the ``push rules'' (to be described next) will recover this reduction. If not, then 
%% we are not gaining anything if we push the instantiation inwards to get: $\inst{\gamma_1}{\phi}; \gamma_2[\phi/a]; \inst{\gamma_3}{\phi}$. In fact, the 
%% resulting term is bigger than the one we started with! 

\subsubsection{Pushing transitivity down}
Rules \rulename{PushApp}, \rulename{PushAll}, \rulename{PushNth}, and \rulename{PushInst} push uses of transitivity
{\em down} the structure of a coercion term, towards the leaves. These rules aim to reveal more redexes
at the leaves, that will be reduced by the next (and final) set of rules. Notice that rules \rulename{PushInst} and \rulename{PushNth}
impose side conditions on the transitive composition $\gamma_1;\gamma_2$. Without these conditions, the resulting coercion may not be well-formed.
Take $\gamma_1 = \forall a{:}\eta @.@ \refl{T\;a\;a}$ and $\gamma_2 = \forall a{:}\eta @.@ \refl{T\;a\;@Int@}$. It is 
certainly the case that $(\inst{\gamma_1}{@Int@});(\inst{\gamma_2}{@Int@})$ is well formed. However, $\wfco \gamma_1 : \forall a{:}\eta @.@ T\;a\;a \psim \forall a{:}\eta @.@ T\;a\;a$
and $\wfco \gamma_2 : \forall a{:}\eta @.@ T\;a\;@Int@ \psim \forall a{:}\eta @.@ T\;a\;@Int@$, and hence $\inst{(\gamma_1;\gamma_2)}{@Int@}$ is not well-formed. A similar argument 
applies to rule \rulename{PushNth}. 

\begin{figure}[t]\small
\[\begin{array}{c}
%% \text{Leaf rules} \\ \\ 
\prooftree 
   \Delta \wfco c : \tau \psim \upsilon
  \minusv{VarSym} 
   \Delta \vdash c ; \sym{c} \rsa{} \refl{\tau}
  \twiddleiv
   \Delta \wfco c : \tau \psim \upsilon
  \minusv{SymVar} 
   \Delta \vdash \sym{c} ; c \rsa{} \refl{\upsilon}
  \twiddlev 
          (C\, \ol{(a{:}\eta)} : \tau \psim \upsilon) \in \Gamma \quad  \as \subseteq ftv(\upsilon)
  \minusv{AxSym} 
   \begin{array}{l} 
   \Delta \vdash C\;\gammas_1;\sym{(C\;\gammas_2)} \rsa{} \\ 
   \qquad\quad \lifting{\as \mapsto \ol{\gamma_1;\sym{\gamma_2}}}{\tau}
   \end{array}
  \twiddleiv 
           (C\,\ol{(a{:}\eta)} : \tau \psim \upsilon) \in \Gamma \quad \as \subseteq ftv(\tau)
  \minusv{SymAx}
   \begin{array}{l} 
   \Delta \vdash \sym{(C\;\gammas_1)};C\;\gammas_2 \rsa{} \\ 
   \qquad\quad \lifting{\as \mapsto \ol{\sym{\gamma_1};\gamma_2}}{\upsilon}
   \end{array}
  \twiddlev
  \begin{array}{c} 
      (C\,\ol{(a{:}\eta)} : \tau \psim \upsilon) \in \Gamma \\ \as \subseteq ftv(\upsilon) \quad
       nontriv(\delta) \\ \delta = \lifting{\as \mapsto\gammas_2}{\upsilon}
  \end{array}
  \minusv{AxSuckR}
    \Delta \vdash (C\;\gammas_1) ; \delta \rsa{} C\;\ol{\gamma_1{;}\gamma_2}
  \twiddleiv 
  \begin{array}{c} 
      (C \ol{(a{:}\eta)} : \tau \psim \upsilon) \in \Gamma \\ \as \subseteq ftv(\tau) \quad
       nontriv(\delta) \\ \delta = \lifting{\as \mapsto\gammas_1}{\tau} \vspace{2pt}
  \end{array}
  \minusv{AxSuckL} 
    \Delta \vdash \delta ; (C\;\gammas_2) \rsa{} C\;\ol{\gamma_1{;}\gamma_2}
  \twiddlev
  \begin{array}{c} 
      (C\, \ol{(a{:}\eta)} : \tau \psim \upsilon) \in \Gamma \quad \as \subseteq ftv(\tau) \\
       nontriv(\delta) \quad \delta = \lifting{\as \mapsto\gammas_2}{\tau} \vspace{2pt}
  \end{array}
  \minusv{SymAxSuckR} 
    \Delta \vdash \sym{(C\;\gammas_1)} ; \delta \rsa{} \sym{(C\;\ol{\sym{\gamma_2}{;}\gamma_1})}
  \twiddlev 
  \begin{array}{c} 
      (C\,\ol{(a{:}\eta)} : \tau \psim \upsilon) \in \Gamma \quad \as \subseteq ftv(\upsilon) \\
       nontriv(\delta) \quad \delta = \lifting{\as \mapsto\gammas_1}{\upsilon}
  \end{array}
  \minusv{SymAxSuckL}
    \Delta \vdash \delta ; \sym{(C\;\gammas_2)} \rsa{} \sym{(C\;\ol{\gamma_2{;}\sym{\gamma_1}})}
\endprooftree
\end{array}\]\caption{Coercion simplification (II)}\label{fig:optimization2}
\end{figure}

\subsubsection{Leaf reactions}
When transitivity and symmetry have been pushed as low as possible, new redexes may appear, for which we introduce
rules \rulename{VarSym}, \rulename{SymVar}, \rulename{AxSym}, \rulename{SymAx}, \rulename{AxSuckR}, \rulename{AxSuckL}, 
\rulename{SymAxSuckR}, \rulename{SymAxSuckL}. (Figure~\ref{fig:optimization2})
\begin{itemize*} 
  \item Rules \rulename{VarSym} and \rulename{SymVar} are entirely straightforward: a coercion variable (or its symmetric coercion) meets its
symmetric coercion (or the variable) and the result is the identity. 

  \item Rules \rulename{AxSym} and \rulename{SymAx} are more involved. Assume that the axiom $(C\;(\ol{a{:}\eta}) {:} \tau \psim \upsilon) \in \Gamma$, and a 
  well-formed coercion of the form: $C\;\gammas_1; \sym{(C\;\gammas_2)}$. Moreover $\Delta \wfco \gammas_1 : \sigmas_1 \psim \phis_1$ and $\Delta \wfco \gammas_2 : \sigmas_2 \psim \phis_2$.
Then we know that $\Delta \wfco C\;\gammas_1; \sym{(C\;\gammas_2)} : \tau[\sigmas_1/\as] \psim \tau[\sigmas_2/\as]$. Since the composition is 
well-formed, it must be the case that $\upsilon[\phis_1/\as] = \upsilon[\phis_2/\as]$. If $\as \subseteq ftv(\upsilon)$ then it must be $\phis_1 = \phis_2$. Hence,
the pointwise composition $\ol{\gamma_1;\sym{\gamma_2}}$ is well-formed and of type $\sigmas_1 \psim \sigmas_2$. Consequently, we may replace the original coercion 
with the {\em lifting} of $\tau$ over a substitution that maps $\as$ to $\ol{\gamma_1;\sym{\gamma_2}}$: $\lifting{\as \mapsto \ol{\gamma_1;\sym{\gamma_2}}}{\tau}$.

What is this lifting operation, of a substitution from type variables to coercions, over a type? 
Its result is a new coercion, and the definition of the operation is given in Figure~\ref{fig:lifting}. 
The easiest way to understand it is by its effect on a type:
\begin{lemma}[Lifting]
If $\Delta,(a{:}\eta) \wfty \tau : \eta$ and
$\Delta \wfco \gamma : \sigma \sim \phi$ such that $\Delta \wfty \sigma : \eta$ and $\Delta \wfty \phi : \eta$, 
then $\Delta \wfco \lifting{a \mapsto \gamma}{\tau} : \tau[\sigma/a] \psim \tau[\phi/a]$
\end{lemma}
Notice that we have made sure that lifting pulls reflexivity as high as possible in the syntax tree -- the only 
significance of this on-the-fly normalization was that it appeared to simplify the argument we have given for termination of 
coercion normalization.

\begin{figure}\small
\[\begin{array}{l}
\ruleform{\lifting{a\mapsto\gamma}{\tau} = \gamma'} \\ \\
\begin{array}{lcl}
\lifting{a \mapsto \gamma}{a}    & = & \gamma \\
\lifting{a \mapsto \gamma}{b}    & = & \refl{b} \\ 
\lifting{a \mapsto \gamma}{H}    & = & \refl{H} \\
\lifting{a \mapsto \gamma}{F}    & = & \refl{F} \\
\lifting{a \mapsto \gamma}{\tau_1\;\tau_2} & = & 
   \left\{\begin{array}{l} 
           \refl{\phi_1\;\phi_2} \text{ when } \lifting{a\mapsto\gamma}{\tau_i} = \refl{\phi_i} \\
           (\lifting{a \mapsto \gamma}{\tau_1})\;(\lifting{a\mapsto\gamma}{\tau_2}) \text{ otherwise }
         \end{array}\right.   \\
\lifting{a\mapsto\gamma}{\forall b{:}\eta @.@ \tau} & = &  
   \left\{\begin{array}{l} 
           \refl{\forall a{:}\eta @.@ \phi} \text{ when } \lifting{a\mapsto\gamma}{\tau} = \refl{\phi} \\
           \forall b{:}\eta @.@ (\lifting{a\mapsto\gamma}{\tau}) \text{ otherwise} \quad (b \notin ftv(\gamma), b \neq a)
         \end{array}\right.
\end{array}
%% \begin{array}{llcl} 
%%         & \tcase{\tcast{K\;\taus\;\gammas\;\es}{\gamma}}{p \to u} & \rightsquigarrow & \tcase{K\;\taus'\;\gammas'\;\es'}{p \to u} \\ 
%%                    & & & \hspace{-4pt}\begin{array}{lll} 
%%                                   \text{when } \\ %% & \cval(K\;\taus\;\phis\;\es)  \\ 
%%                                   & \wfco \gamma : T\;\taus \psim T\;\taus' \\ 
%%                                   & K{:}\forall\ol{a{:}\eta} @.@ \ol{\sigma_1 \psim \sigma_2} \to \sigmas \to T\;\as \in \Gamma \\ 
%%                                   & e_i' = \tcast{e_i}{\lifting{\as \mapsto \deltas}{\sigma_i[\phis/\cs]}} \\
%%                                   & \delta_j = \nth{j}{\gamma} \\ 
%%                                   & \gamma_j' = \ldots 
%%                      \end{array}
%% \end{array}
\end{array} \]
\caption{Lifting}\label{fig:lifting}
\end{figure}

Returning to rules \rulename{AxSym} and \rulename{SymAx}, we stress that the side condition is essential for the rule to be sound. Consider the following example: 
\[  C (a{:}\star): F\;[a] \psim  @Int@  \in \Gamma  \]
Then $(C\;\refl{@Int@});\sym{(C\;\refl{@Bool@})}$ is well-formed and of 
type $F\;[@Int@] \psim F\;[@Bool@]$, but $\refl{F}\;(\refl{@Int@};\sym{\refl{@Bool@}})$ is not well-formed!
Rule \rulename{SymAx} is symmetric and has a similar soundness side condition on the free variables of $\tau$ this time. 

  \item The rest of the rules deal with the case when an axiom meets a lifted type -- the reaction swallows the lifted type
        inside the axiom application. For instance, here is rule \rulename{AxSuckR}: 
        \[\small\prooftree
        \begin{array}{c} 
          (C\;(\ol{a{:}\eta}) {:} \tau \psim \upsilon) \in \Gamma \quad \as \subseteq ftv(\upsilon) \\
          nontriv(\delta) \quad \delta = \lifting{\as \mapsto\gammas_2}{\upsilon} \vspace{2pt}
        \end{array}
        \minusv{AxSuckR}
        \Delta \vdash (C\;\gammas_1) ; \delta \rsa{} C\;\ol{\gamma_1{;}\gamma_2}
        \endprooftree\]
        This time let us assume that $\Delta \wfco \gammas_1 : \sigmas_1 \psim \phis_1$. Consequently
        $\Delta \wfco C\;\gammas_1 : \tau[\sigmas_1/\as] \psim \upsilon[\phis_1/\as]$. Since $\as \subseteq ftv(\upsilon)$ it 
        must be that $\Delta \wfco \gammas_2 : \phis_1 \psim \phis_3$ for some $\phis_3$ and we can 
        pointwise compose $\ol{\gamma_1{;}\gamma_2}$ to get coercions between $\sigmas_1 \psim \phis_3$. 
        The resulting coercion $C\;\ol{\gamma_1{;}\gamma_2}$ is well-formed and of type $\tau[\sigmas_1/\as] \psim \upsilon[\phis_3/\as]$. 
        Rules \rulename{AxSuckL}, \rulename{SymAxSuckL}, and \rulename{SymAxSuckR} involve a similar reasoning.

        The side condition $nontriv(\delta)$ is not restrictive in any way -- it merely requires that $\delta$ contains some variable 
        $c$ or axiom application. If not, then $\delta$ can be converted to reflexivity:
        \begin{lemma}\label{lem:coherence} 
        If $\wfco \delta : \sigma{\psim}\phi$ and $\lnot nontriv(\delta)$, then $\delta {\longrightarrow^{*}} \refl{\phi}$.
        \end{lemma}
        Reflexivity, when transitively composed with any other coercion, is eliminable via \rulename{ReflElimL/R} or and consequently the side condition is not preventing any 
        reactions from firing. It will, however, be useful in the simplification termination proof in Section~\ref{ssect:termination}.
\end{itemize*}

The purpose of rules \rulename{AxSuckL/R} and \rulename{SymAxSuckL/R} is to eliminate intermediate coercions in a
big transitive composition chain, to give the opportunity to an axiom to meet its symmetric version and react 
with rules \rulename{AxSym} and \rulename{SymAx}. In fact this rule is {\em precisely} what we need for the 
impressive simplifications from Section~\ref{ssect:large}. Consider that example again:
\[\begin{array}{lrll}
 \gamma_5 & = &  \gamma_2;\gamma_3;\gamma_4 \\
          & = &  \sym{(C_N\;\refl{\tau_1})};(\refl{N}\;\gamma_1);(C_N\;\refl{\tau_2})    & (\rulename{AxSucL} \text{ with } \delta := (\refl{N}\;\gamma_1)) \\
          & \longrightarrow & \sym{(C_N\;\refl{\tau_1})};(C_N\;(\gamma_1;\refl{\tau_2})) & (\rulename{ReflElimR} \text{ with } \gamma := \gamma_1, \phi := \tau_2) \\
          & \longrightarrow & \sym{(C_N\;\refl{\tau_1})};(C_N\;\gamma_1)                 & (\rulename{SymAx}) \\
          & \longrightarrow & \refl{\to}\;(\refl{\tau_1};\gamma_1)\;\refl{@Int@}        & (\rulename{ReflElimL} \text{ with } \phi := \tau_1,\gamma := \gamma_1) \\
          & \longrightarrow & \refl{\to}\;\gamma_1\;\refl{@Int@} 
\end{array}\] 

Notably, rules \rulename{AxSuckL/R} and \rulename{SymAxSuckL/R} generate
axiom applications of the form $C\;\gammas$ (with a coercion as argument).
In our previous papers, the syntax of axiom applications was $C\;\taus$, with \emph{types}
as arguments.  But we need the additional generality to allow coercions rewriting to
proceed without getting stuck.

%% which we now give in mathematical notation. The 
%% relevant axioms are:
%% \[\begin{array}{lll} 
%%     C_n  : \forall a{:}\star \to \star @.@ N\;a \psim \forall xy @.@ a\;x \to a\;y  & \in & \Gamma \\ 
%%     C_f  : F\;() \psim Maybe                                 & \in & \Gamma 
%% \end{array}\]
%% The coercion term is: 
%% \[  \nth{2}{(
%%          \inst{(\inst{(\sym{C_n\;\refl{Maybe}};\refl{N}\;(\sym{C_f});C_n\;\refl{F\;()})}{x_{a}})}{y_{a}})} \]

%% Its simplification is given in Figure~\ref{fig:optimization-example}. 
%% Notably, rules \rulename{AxSuckL/R} and \rulename{SymAxSuckL/R} rely 
%% on axiom applications be of the form $C\;\gammas$ instead of the simpler 
%% $C\;\taus$ found in previous FC papers.

%% \begin{figure*}\small
%% \[\begin{array}{ll}
%%    \nth{2}{(
%%            \inst{(\inst{(\sym{C_n\;\refl{Maybe}};\highlight{\refl{N}\;(\sym{C_f});C_n\;\refl{F\;()}})}{x_{a}})}{y_{a}})} \vspace{3pt} \\
%% (\rulename{AxSuckL}) \rsa{} \vspace{3pt}\\
%%    \nth{2}{(
%%            \inst{(\inst{\highlight{(\sym{C_n\;\refl{Maybe}};C_n\;((\sym{C_f});\refl{F\;()}))}}{x_{a}})}{y_{a}})} \vspace{3pt} \\ 
%% (\rulename{SymAx})    \rsa{} \vspace{3pt}\\ 
%%      \nth{2}{(
%%          \inst{(\inst{(\forall xy. (\sym{\refl{Maybe}};\sym{C_f};\refl{F\;()})\;\refl{x}\;\refl{\to}\;
%%                             (\sym{\refl{Maybe}};\sym{C_f};\refl{F\;()})\;\refl{y})}{x_{a}})}{y_{a}})} \vspace{3pt}\\
%% (\rulename{ReflElimL},\rulename{ReflElimR},\rulename{SymRefl}) \rsa{}^{*} \vspace{3pt}\\
%%      \nth{2}{(
%%          \inst{(\inst{(\forall x y @.@ (\sym{C_f})\;\refl{x}\;\refl{\to}\;
%%                             (\sym{C_f})\;\refl{y})}{x_{a}})}{y_{a}})} \vspace{3pt}\\
%% (\rulename{RedInstCo}) \rsa{}^{*} \quad
%%      \nth{2}{((\sym{C_f})\;\refl{x_a}\;\refl{\to}\;(\sym{C_f})\;\refl{y_a})} \quad 
%% (\rulename{RedNth}) \rsa{} \quad
%%      (\sym{C_f})\;\refl{y_a} 
%% \end{array}\]
%% \caption{Simplification example}\label{fig:optimization-example}
%% \end{figure*}

\section{Coercion simplification in GHC}\label{ssect:ghc}

To assess the usefulness of coercion simplification we added it to GHC. 
For Haskell programs that make no use of GADTs or type families, the
effect will be precisely zero, so we took measurements on two bodies of code.
First, our regression suite of 151 tests for GADTs and type families; these are
all very small programs.  Second, the @Data.Accelerate@ library that we know makes use
of type families \cite{chakravarty+:accelerate}. This library consists
of 18 modules, containing 8144 lines of code.

We compiled each of these programs with and without coercion simplification, 
and measured the percentage reduction in size of the coercion terms with
simplification enabled.  This table shows the minimum, maximum, and 
aggregate reduction, taken over the 151 tests and 18 modules respectively.
The ``aggregate reduction'' is obtained by combining all the programs
in the group (testsuite or @Accelerate@) into one giant ``program'', and computing
the reduction in coercion size.
$$
\begin{array}{rrr}
& \text{Testsuite} & \text{Accelerate} \\
\text{Minimum} & -97\% & -81\%\\
\text{Maximum} & +14\% & 0\% \\
\text{\bf Aggregate} & {\bf -58\% } & {\bf -69\%}
\end{array}
$$
There is a substantial aggregate decrease of 58\% in the testsuite
and 69\% in @Accelerate@, with a massive 97\% decrease
in special cases. These special cases should not be taken lightly: 
in one program the types and coercions taken together were five times
bigger than the term they decorated; after simplification they were ``only''
twice as big.  The coercion simplifier makes the compiler less vulnerable to
falling off a cliff.

Only one program showed an increase in coercion size, of 14\%, which turned out to be the
effect of this rewrite:
$$ \sym{(C ;  D)} \quad \longrightarrow \quad (\sym{D});(\sym{C}) $$
Smaller coercion terms make the 
compiler faster, but the normalization algorithm itself consumes some time.
However, the effect on compile time is barely measurable (less than 1\%), and we 
do not present detailed figures.

Of course none of this would matter if coercions were always tiny, so that they
took very little space in the first place.  And indeed that is often the case.
But for programs that make heavy use of type functions, un-optimised coercions 
can dominate compile time. For example, the @Accelerate@ library makes
heavy use of type functions.  The time and memory consumption of compiling
all 21 modules of the library are as follows:
$$
\begin{array}{lrrr}
                                     & \text{Compile time}  & \text{Memory allocated} & \text{Max residency} \\
\text{With coercion optimisation}    & 68s           & 31\, Gbyte & 153\, Mbyte \\
\text{Without coercion optimisation} & 291s          & 51\, Gbyte & 2,000\, Mbyte 
\end{array}
$$
As you can see, the practical effects can be extreme; the cliff is very real.

\section{Termination and confluence}\label{ssect:termination}
\newcommand{\ps}{\ol{p}}

We have demonstrated the effectiveness of the algorithm in practice, but we must also 
establish termination. This is important, since it would not be acceptable
for a compiler to loop while simplifying a coercion, no matter what axioms are declared by users.
Since the rules fire non-deterministically, and some of the rules (such as \rulename{RedInstCo} or \rulename{AxSym})
create potentially larger coercion trees, termination is not obvious. 

\subsection{Termination}

\begin{figure*}\small
\[\begin{array}{c} 
 \begin{array}{lcl} 
      \multicolumn{3}{l}{\text{Axiom polynomial}} \\ 
      p(\sym{\gamma})    & = & p(\gamma) \\ 
      p(C\;\gammas) & = & z \cdot \Sigma p(\gamma_i) + z + 1 \\ 
      p(c)          & = & 1 \\ 
      p(\gamma_1;\gamma_2)     & = & p(\gamma_1) + p(\gamma_2) + p(\gamma_1)\cdot p(\gamma_2) \\ 
      p(\refl{\phi})   & = & 0 \\ 
      p(\nth{k}{\gamma}) & = & p(\gamma) \\ 
      p(\inst{\gamma}{\phi}) & = & p(\gamma) \\ 
      p(\gamma_1\;\gamma_2)    & = & p(\gamma_1) + p(\gamma_2) \\
      p(\forall a{:}\eta @.@ \gamma) & = & p(\gamma)
  \end{array}
  \begin{array}{lcl} 
      \multicolumn{3}{l}{\text{Coercion weight}} \\ 
      w(\sym{\gamma})      & = & w(\gamma) \\ 
      w(C\;\gammas)   & = & \Sigma w(\gamma_i) + 1 \\ 
      w(c)            & = & 1 \\ 
      w(\gamma_1;\gamma_2)       & = & 1 + w(\gamma_1) + w(\gamma_2) \\ 
      w(\refl{\phi})     & = & 1 \\ 
      w(\nth{k}{\gamma})   & = & 1 + w(\gamma) \\ 
      w(\inst{\gamma}{\phi})  & = & 1 + w(\gamma) \\ 
      w(\gamma_1\;\gamma_2)      & = & 1 + w(\gamma_1) + w(\gamma_2) \\ 
      w(\forall a{:}\eta @.@ \gamma) & = & 1 + w(\gamma)
  \end{array} \\ 
\begin{array}{lcl} 
      \multicolumn{3}{l}{\text{Symmetry weight}} \\ 
      sw(\sym{\gamma})      & = & w(\gamma) + sw(\gamma) \\
      sw(C\;\gammas)   & = & \Sigma sw(\gamma_i) \\ 
      sw(c)            & = & 0 \\ 
      sw(\gamma_1;\gamma_2)       & = & sw(\gamma_1) + sw(\gamma_2) \\ 
      sw(\refl{\phi})     & = & 0 \\ 
      sw(\nth{k}{\gamma})   & = & sw(\gamma) \\ 
      sw(\inst{\gamma}{\phi})  & = & sw(\gamma) \\ 
      sw(\gamma_1\;\gamma_2)      & = & sw(\gamma_1) + sw(\gamma_2) \\
      sw(\forall a{:}\eta @.@ \gamma) & = & sw(\gamma)
     \end{array}
\end{array}\]\caption{Metrics on coercion terms}\label{fig:metrics}
\end{figure*}

To formalize a termination argument, we introduce several definitions in Figure~\ref{fig:metrics}. 
The {\em axiom polynomial} of a coercion over a distinguished variable $z$, $p(\cdot)$, returns a polynomial with natural number coefficients that can be compared
to any other polynomial over $z$. The {\em coercion weight} of a coercion is defined as the function $w(\cdot)$ and 
the {\em symmetry weight} of a coercion is defined with the function $sw(\cdot)$ in 
         Figure~\ref{fig:metrics}. Unlike the polynomial and coercion weights of a coercion, 
         $sw(\cdot)$ does take symmetry into account.
Finally, we will also use the {\em number of coercion applications and coercion $\forall$-introductions}, denoted with $intros(\cdot)$ in what follows.

Our termination argument comprises of the lexicographic left-to-right ordering of:
\[  \mu(\cdot) = \langle p(\cdot),w(\cdot),intros(\cdot),sw(\cdot)\rangle \]
We will show that each of the $\rsa{}$ reductions reduces this tuple.
For this to be a valid termination argument for $(\longrightarrow)$ we need two more facts about 
{\em each} component measure, namely that (i)~$(=)$ and $(<)$ are preserved under arbitrary contexts, 
and (ii)~each component is invariant with respect to the associativity of $(;)$.

\begin{lemma} If $\Delta \wfco \gamma_1 : \tau \psim \sigma$ and $\gamma_1 \approx \gamma_2$ modulo associativity 
of $(;)$, then $p(\gamma_1) = p(\gamma_2)$, $w(\gamma_1) = w(\gamma_2)$, $intros(\gamma_1) = intros(\gamma_2)$, and $sw(\gamma_1) = sw(\gamma_2)$.
\end{lemma} 
\vspace{-10pt}\begin{proof} This is a simple inductive argument, the only interesting case is the case for 
$p(\cdot)$ where the reader can calculate that $p(\gamma_1;(\gamma_2;\gamma_3)) = p((\gamma_1;\gamma_2);\gamma_3)$ and by induction we are done. 
\end{proof}

\begin{lemma} If $\Gamma,\Delta \wfco \gamma_i : \tau \psim \sigma$ (for $i=1,2$) and $p(\gamma_1) < p(\gamma_2)$ then 
$p(\G[\gamma_1]) < p(\G[\gamma_2])$ for any $\G$ with $\Gamma \wfco \G[\gamma_i] : \phi \psim \phi'$.
Similarly if we replace $(<)$ with $(=)$. 
\end{lemma}
\vspace{-10pt}\begin{proof} By induction on the shape of $\G$. The only interesting case is the transitivity case 
again. Let $\G = \gamma ; \G'$. Then $p(\gamma;\G'[\gamma_1]) = p(\gamma) + p(\G'[\gamma_1]) + p(\gamma)\cdot p(\G'[\gamma_1])$ whereas 
$p(\gamma;\G'[\gamma_2]) = p(\gamma) + p(\G'[\gamma_2]) + p(\gamma)\cdot p(\G'[\gamma_2])$. Now, either $p(\gamma) = 0$, in which case we are done
by induction hypothesis for $\G'[\gamma_1]$ and $\G'[\gamma_2]$, or $p(\gamma) \neq 0$ in which case again induction 
hypothesis gives us the result since we are multiplying $p(\G'[\gamma_1])$ and $p(\G'[\gamma_2])$ by the same polynomial.
The interesting ``trick'' is that the polynomial for transitivity contains both the product 
of the components {\em and} their sum (since product alone is not preserved by contexts!).
\end{proof}
\begin{lemma} If $\Gamma,\Delta \wfco \gamma_i : \tau \psim \sigma$ and $w(\gamma_1) < w(\gamma_2)$ then 
$w(\G[\gamma_1]) < w(\G[\gamma_2])$ for any $\G$ with $\Gamma \wfco \G[\gamma_i] : \phi \psim \phi'$.
Similarly if we replace $(<)$ with $(=)$.
\end{lemma}

\begin{lemma} If $\Gamma,\Delta \wfco \gamma_i : \tau \psim \sigma$ and $intros(\gamma_1) < intros(\gamma_2)$ then 
$intros(\G[\gamma_1]) < intros(\G[\gamma_2])$ for any $\G$ with $\Gamma \wfco \G[\gamma_i] : \phi \psim \phi'$.
Similarly if we replace $(<)$ with $(=)$. 
\end{lemma}


\begin{lemma} If $\Gamma,\Delta \wfco \gamma_i : \tau \psim \sigma$, $w(\gamma_1) \leq w(\gamma_2)$, and $sw(\gamma_1) < sw(\gamma_2)$ then 
$sw(\G[\gamma_1]) < sw(\G[\gamma_2])$ for any $\G$ with $\Gamma \wfco \G[\gamma_i] : \phi \psim \phi'$.
\end{lemma}
\vspace{-10pt}\begin{proof} The only interesting case is when $\G = \sym{\G'}$ and hence we have that 
$sw(\G[\gamma_1]) = sw(\sym{\G'[\gamma_1]}) = w(\G'[\gamma_1]) + sw(\G'[\gamma_1])$. 
Similarly $sw(\G[\gamma_2]) = w(\G'[\gamma_2]) + sw(\G'[\gamma_2])$. By the precondition for the weights and induction 
hypothesis we are done. The precondition on the weights is not restrictive, since 
$w(\cdot)$ has higher precedence than $sw(\cdot)$ inside $\mu(\cdot)$.
\end{proof}

The conclusion is the following theorem.
\begin{theorem}
If $\gamma \approx \G[\gamma_1]$ modulo associativity of $(;)$ and $\Delta \wfco \gamma_1 : \sigma \psim \phi$, and 
$\Delta \vdash \gamma_1 \rsa{} \gamma_2$ such that $\mu(\gamma_2) < \mu(\gamma_1)$, it is the case that $\mu(\G[\gamma_2]) < \mu(\gamma)$.
\end{theorem}

\begin{corollary} 
$(\longrightarrow)$ terminates on well-formed coercions if each of the $\rsa{}$ transitions reduces $\mu(\cdot)$. 
\end{corollary}

Note that often the term rewrite literature requires similar
conditions (preservation under contexts and associativity), but also
{\em stability under substitution} (e.g. see~\cite{Baader:1998:TR:280474},
Chapter 5). In our setting, variables are essentially treated as
constants and this is the reason that we do not rely on stability
under substitutions. For instance the rule \rulename{ReflElimR}
$\Delta |- \gamma;\refl{\phi} \rsa{} \gamma$ is {\em not} expressed as
$\Delta |- c;\refl{\phi} \rsa{} c$, as would be customary in a more
traditional term-rewrite system presentation.

We finally show that indeed each of the $\rsa{}$ steps reduces $\mu(\cdot)$.

\begin{theorem}[Termination]
If $\Delta \wfco \gamma_1 : \sigma \psim \phi$ and $\Delta \vdash \gamma_1 \rsa{} \gamma_2$ then $\mu(\gamma_2) < \mu(\gamma_1)$. 
\end{theorem}
\vspace{-10pt}\begin{proof}
It is easy to see that the reflexivity rules, the symmetry rules, the reduction rules, and the 
$\eta$-rules preserve or reduce the polynomial component $p(\cdot)$. The same is true for the push rules 
but the proof is slightly more interesting. Let us consider \rulename{PushApp}, and let us write
$p_i$ for $p(\gamma_i)$. We have that 
$p((\gamma_1\;\gamma_2);(\gamma_3\;\gamma_4)) = p_1 + p_2 + p_3 + p_4 + p_1p_3 + p_2p_3 + p_1p_4 + p_2p_4$. On 
the other hand $p((\gamma_1;\gamma_3)\;(\gamma_2;\gamma_4)) = p_1 + p_3 + p_1p_3 + p_2 + p_4 + p_2p_4$ which is a smaller
or equal polynomial than the left-hand side polynomial.
Rule \rulename{PushAll} is easier. Rules \rulename{PushInst} and \rulename{PushNth} have exactly the same polynomials on the left-hand and 
the right-hand side so they are ok. Rules \rulename{VarSym} and \rulename{SymVar} reduce $p(\cdot)$. 
The interesting bit is with rules \rulename{AxSym}, \rulename{SymAx}, and \rulename{AxSuckR/L} 
and \rulename{SymAxSuckR/L}. We will only show the cases for \rulename{AxSym} and \rulename{AxSuckR} 
as the rest of the rules involve very similar calculations: 
\begin{itemize*}
  \item Case \rulename{SymAx}. We will use the notational convention $\ps_1$ for $p(\gammas_1)$ (a vector of polynomials)
   and similarly $\ps_2$ for $p(\gammas_2)$. Then the left-hand side polynomial is:
   \[\begin{array}{l} 
      (z\Sigma\ps_1{+}z{+}1) + (z\Sigma\ps_2{+}z{+}1) + \\ 
      \quad\quad\quad\quad\quad\quad (z\Sigma\ps_1{+}z{+}1)\cdot(z\Sigma\ps_2{+}z{+}1) = \\
      (z^2{+}2z)\Sigma\ps_1 + (z^2{+}2z)\Sigma\ps_2 + z^2\Sigma\ps_1\Sigma\ps_2 + (z^2{+}4z{+}3)
   \end{array}\] 
  For the right-hand side polynomial we know that each $\gamma_{1i};\sym{\gamma_{2i}}$ will have polynomial
  $p_{1i}+p_{2i}+p_{1i} p_{2i}$ and it cannot be repeated inside the lifted type more than a finite
  number of times (bounded by the maximum number of occurrences of a type variable from $\as$ in
  type $\tau$), call it $k$. Hence the right-hand side polynomial is smaller or equal to: 
  \[\begin{array}{ll}
       k\Sigma\ps_1 + k\Sigma\ps_2 + k\Sigma(p_{1i}p_{2i}) \leq 
       k\Sigma\ps_1 + k\Sigma\ps_2 + k\Sigma\ps_1\Sigma\ps_2   
  \end{array}\] 
  But that polynomial is strictly smaller than the left-hand side polynomial, hence we are done.
  \item Case \rulename{AxSuckR}. In this case the left-hand side polynomial is going to be greater
        or equal to (because of reflexivity inside $\delta$ and because some of the $\as$ variables may 
        appear more than once inside $\upsilon$ it is not exactly equal to) the following: 
        \[\begin{array}{l} 
           (z\Sigma\ps_1+z+1) + \Sigma\ps_2 + (z\Sigma\ps_1+z+1)\Sigma\ps2 = \\ 
           \quad\quad\quad\quad\quad z\Sigma\ps_1\Sigma\ps_2 + z\Sigma\ps_1 + z\Sigma\ps_2 + 2\Sigma\ps_2 + z + 1 
          \end{array}\]
        On the other hand, the right-hand side polynomial is: 
$$
           z\Sigma(p_{1i}+p_{2i}+p_{1i}p_{2i})+z+1 \, \leq \,   
           z\Sigma\ps_1+z\Sigma\ps_2+z\Sigma\ps_1\Sigma\ps_2+z+1
$$
        We observe that there is a difference of $2\Sigma\ps_2$, but we know that 
        $\delta$ satisfies $nontriv(\delta)$, and consequently there must exist some variable or 
        axiom application inside one of the $\gammas_2$. Therefore, $\Sigma\ps_2$ is 
        {\em non-zero} and the case is finished.
\end{itemize*}
It is the arbitrary copying of coercions $\gammas_1$ and $\gammas_2$ in rules \rulename{AxSym} and \rulename{SymAx} 
that prevents simpler measures that only involve summation of coercions for axioms or transitivity. Other reasonable 
measures such as the height of transitivity uses from the leaves would not be preserved from contexts, 
due to \rulename{AxSym} again.

So far we've shown that all rules but the axiom rules preserve the polynomials, and the axiom rules 
reduce them. We next show that in the remaining rules, some other component reduces, lexicographically.
Reflexivity rules reduce $w(\cdot)$. Symmetry rules preserve $w(\cdot)$ and $intros(\cdot)$ 
but reduce $sw(\cdot)$. Reduction rules and $\eta$-rules reduce $w(\cdot)$. 
Rules \rulename{PushApp} and \rulename{PushAll} preserve or reduce $w(\cdot)$ but certainly 
reduce $intros(\cdot)$. Rules \rulename{PushInst} and \rulename{PushNth} reduce $w(\cdot)$. 
\end{proof}
We conclude that $(\longrightarrow)$ terminates.

\subsection{Confluence}

Due to the arbitrary types of axioms and coercion variables in the context, we do not expect 
confluence to be true. Here is a short example that demonstrates the lack of 
confluence; assume we have the following in our context:
\[\begin{array}{lcl} 
C_1 \, (a{:}\star \to \star) : F\;a \psim a & \\ 
C_2 \, (a{:}\star \to \star) : G\;a \psim a & 
\end{array}\]
Consider the coercion:
\[      (C_1\;\refl{\sigma});\sym{(C_2\;\refl{\sigma})}  \] 
of type $F\;\sigma \psim G\;\sigma$. In one reduction possibility, using 
rule \rulename{AxSuckR}, we may get 
\[ C_1\;(\sym{(C_2\;\refl{\sigma})}) \] 
In another possibility, using \rulename{SymAxSuckL}, we 
may get 
\[ \sym{(C_2\;(\sym{(C_1\;\refl{\sigma})}))} \]
Although the two normal forms are different, it is unclear if one of them is ``better'' than the other. 

Despite this drawback, confluence or syntactic characterization of normal forms is, for our purposes, 
of secondary importance (if possible at all for open coercions in such an under-constrained problem!), 
since we never reduce coercions for the purpose of comparing their normal forms. That said, we acknowledge 
that experimental results may vary with respect to the actual evaluation strategy, but we do not expect 
wild variations.


%% \begin{array}{l} 
%% \nth{2}{(
%%          \inst{(\inst{(\sym{N\;\refl{Maybe}};C\;\sym{TF};N\;(F\;()))}{x_{a}})}{y_{a}})} \\ 
%% \rightsquigarrow 




%% axiom  N a :: C a ~ forall xy. a x -> a y
%% axiom  TF   :: F () ~ Maybe
%%   nth 2 
%%     (inst
%%        (inst
%%           (trans
%%              (sym 
%%                 (N Maybe)
%%              ) 
%%              (trans (C (sym TF()))
%%                     (N (F ()))
%%              ))
%%           xabL)
%%        yabM)
%%   :: Maybe yabM  ~ F () yabM
%% Here is another motivating example from GHC: \dv{Add example.}







%% \section{Discussion}\label{s:discuss}
%% \dv{Is there anything we want to write here, at all?}

\section{Related and future work}\label{s:related}

%% \subsection{Coercion erasure}
%% There is a substantial volume of related work on proof erasure in the
%% context of dependent type theory. Our method for sound, runtime, but
%% zero-cost equality proof terms lies in the middle ground between two
%% other general methodologies.

%% \paragraph{Type-based erasure}
%%  On the one hand, Coq~\cite{coq} uses a {\em
%% type-based} erasure process by introducing a special universe for
%% propositions, {\em Prop}.  Terms whose type lives in {\em
%% Prop} are erased even when they are applications of functions
%% (lemmas) to computational terms. This is sound since in Coq
%% the computation language is also strongly normalizing. As we have seen,
%% this is not sound in FC. 
%% \paragraph{Irrelevance-based erasure}
%% On the other hand, {\em irrelevance-based} erasure is another
%% methodology proposed in the context of pure type systems and type
%% theory. In the context of Epigram, \cite{DBLP:conf/types/BradyMM03} present an erasure
%% technique where term-level indices of inductive types can be erased
%% even when they are deconstructed inside the body of a function, since
%% values of the indexed inductive datatype will be simultaneously
%% deconstructed and hence the indices are irrelevant for the
%% computation. In the Agda language~\cite{norell:thesis} there exist plans to adopt a similar 
%% irrelevance-based erasure strategy. Other related work~\cite{mishra:erasure,abel:fossacs11} 
%% proposes erasure in the context of PTSs guided with lightweight programmer annotations.

%% Finally, our approach of separating the ``computational part'' of a 
%% proof, which always has to run before we get to a zero-cost ``logical part''
%% is reminiscent of the separation that A-normal forms introduce in refinement 
%% type systems, for instance~\cite{bengtson+:f7}. It is interesting future work to determine
%% whether our treatment of coercions is also applicable
%% to types and hopefully paves the way towards full-spectrum dependent types. 
%% \subsection{Coercion simplification}

Traditionally, work on proof theory is concerned with proof normalization theorems, namely cut-elimination.
Category and proof theory has studied the commutativity of diagrams in {\em monoidal
categories}~\cite{MacLaneS:catwm}, establishing coherence theorems. In our setting Lemma~\ref{lem:coherence}
expresses such a result: any coercion that does not include axioms or free coercion variables is equivalent
to reflexivity. More work on proof theory is concerned with cut-elimination theorems -- in our setting 
eliminating transitivity completely is plainly impossible due to the presence of axioms. 
Recent work on {\em 2-dimensional type theory}~\cite{Licata:2012:CTT:2103656.2103697} provides an equivalence 
relation on equality proofs (and terms), 
which suffices to establish that types enjoy canonical forms. Although that work does not provide an algorithm
for checking equivalence (this is harder to do because of actual computation embedded with isomorphisms), that
definition shares many rules with our normalization algorithm. Finally there is a large literature in associative
commutative rewrite systems~\cite{Dershowitz:1983:AR:1623516.1623594,Bachmair:1985:TOA:6947.6948}.

To our knowledge, most programming languages literature on coercions is not concerned with coercion 
simplification but rather with inferring the placement of coercions in 
source-level programs. Some recent examples are~\cite{luo:coercions} and~\cite{Swamy:2009:TTC:1596550.1596598}. 
A comprehensive study of coercions {\em and their normalization} in programming languages
is that of~\cite{henglein:coercions}, motivated by coercion placement in a language with
{\em type dynamic}. Henglein's coercion language differs to ours in that (i)~coercions there 
are not symmetric, (ii)~do not involve polymorphic axiom schemes and (iii)~may have computational significance.
Unlike us, Henglein is concerned with characterizations of minimal coercions and confluence, 
fixes an equational theory of coercions, and presents a normalization algorithm for that equational theory. 
In our case, in the absence of a denotational semantics for System FC and its coercions, 
such an axiomatization would be no more ad-hoc than the algorithm and 
hence not particularly useful: for instance we could consider adding type-directed
equations like $\Delta \vdash \gamma \rsa{} \refl{\tau}$ when $\Delta \wfco \gamma : \tau \psim \tau$, or other equations 
that only hold in consistent or confluent axiom sets. It is certainly an 
interesting direction for future work to determine whether 
there even exists a maximal syntactic axiomatization of 
equalities between coercions with respect to some denotational semantics 
of System FC.

In the space of typed intermediate languages, {\sf xMLF}\cite{Remy-Yakobowski:xmlf} is
a calculus with coercions that capture {\em instantiation} instead of equality, and which serves 
as target for the {\sf MLF} language. Although the authors are not 
directly concerned with normalization as part of an intermediate 
language simplifier, their translation of the graph-based instantiation 
witnesses does produce {\sf xMLF} normal proofs.

Finally, another future work direction would be to 
determine whether we can encode coercions as $\lambda$-terms, and derive coercion 
simplification by normalization in some suitable $\lambda$-calculus.

%% \dv{Related work seems a bit thin at the moment.}

\paragraph*{Acknowledgments}
Thanks to Tom Schrijvers for early discussions 
and for contributing a first implementation.  We would particularly
like to thank Thomas Str\"{o}der for his insightful and detailed feedback
in the run-up to submitting the final paper.

\bibliographystyle{plain}
\bibliography{fc-normalization-rta}

\end{document}
